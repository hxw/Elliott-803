\documentclass[10pt, a4paper, oneside]{article}

%\usepackage[utf8]{inputenc}
\usepackage{comment}
\usepackage{multicol}
\usepackage[left=2cm, right=2cm, top=2cm, bottom=2cm]{geometry}
\usepackage{fancyhdr}
\usepackage{listings}
\usepackage{titlesec}
\usepackage{amssymb}
\usepackage{mathtools}
\usepackage{capt-of} %% to get the caption
\usepackage{framed}  %% framed multi-line text
\usepackage{scrextend} %% addmargin
\usepackage{soul} %% strikeout: \st{}

\usepackage[xetex]{pict2e}

\usepackage{array}
\newcolumntype{C}{>{\centering\arraybackslash}p{2.2em}}


% ==================================================

% this changes between a two column mode, like the original
% document structure or a larger font single column version
\newif\ifTwoColumn
\ifdefined\TwoColumn
\TwoColumntrue
\else
\TwoColumnfalse
\fi

% ==================================================


\setlength{\unitlength}{1mm}
\setlength{\fboxsep}{0pt}

\usepackage{fontspec}
\setmainfont{Linux Libertine O}

\setlength{\columnsep}{10mm}

\fancyhf{}

\pagestyle{fancy}

% empty footer
\cfoot{   }

\titleformat{\section}[hang]
{\small\bfseries}
{\thesection.}{0.5em}{}

\titleformat{\subsection}[hang]
{\small\bfseries}
{\thesubsection.}{0.5em}{}

% better underline
%\usepackage{contour}
\usepackage{ulem}

\setlength{\ULdepth}{1.0pt}
\newcommand{\myuline}[1]{\uline{#1}}

\renewcommand{\ULdepth}{1.8pt}
%\contourlength{0.8pt}

%\newcommand{\myuline}[1]{%
%  \uline{\phantom{#1}}%
%  \llap{\contour{white}{#1}}%
%}

% to be able top span columns
\newcounter{tempcolnum}

\makeatletter
\ifTwoColumn

\newcommand{\multicolinterrupt}[1]{% Stuff to span both rows
\setcounter{tempcolnum}{\col@number}
\end{multicols}
#1%
\begin{multicols}{\value{tempcolnum}}
}
\else
\newcommand{\multicolinterrupt}[1]{#1}
\fi
\makeatother

% listing font
\lstset{basicstyle=\ttfamily\scriptsize,breaklines=false}

% tt font sizes
\ifTwoColumn
\newcommand{\mytt}[1]{\texttt{\scriptsize #1}}
\else
\newcommand{\mytt}[1]{\texttt{\small #1}}
\fi
\newcommand{\myfntt}[1]{\texttt{\footnotesize #1}}
\newcommand{\mytinytt}[1]{\texttt{\tiny #1}}
\newcommand{\mysctt}[1]{\texttt{\scriptsize #1}}
\newcommand{\mysmalltt}[1]{\texttt{\small #1}}

% roman counter for table rows
\newcounter{RowCounter}
\setcounter{RowCounter}{0}
\newcommand{\nextRow}{\addtocounter{RowCounter}{1}(\roman{RowCounter})}

% user defined functions
\usepackage{amsmath}
\DeclareMathOperator{\Int}{int\ }
\DeclareMathOperator{\Frac}{frac\ }
\DeclareMathOperator{\Modulo}{modulo\ }

%==================================================

\begin{document}

\begin{flushleft}
\textbf{\Huge \fbox{BRUSH} \Large RESEARCH DIVISION} \hfill \normalsize Report No.RDR/102.
\end{flushleft}

\vspace{3cm}

\begin{center}
\myuline{\textbf{H-CODE MANUAL}} \\
\textbf{by} \\
\textbf{D. C. Hogg} \\
\end{center}

\vspace{3cm}

% a horizontal rule
\par\noindent\rule{\textwidth}{0.4pt}\relax

\vspace{3cm}

\begin{center}
\parbox{0.55\textwidth}{%
SUMMARY:- This manual is a description of the facilities
available in the programming system in use on the Brush 803
computer.  It will be supplemented from time to time by additional
sections.  The manual does not attempt to teach programming; it
is a sequel to RDR/100, which introduces H-Code.}
\end{center}

\vspace{3cm}
\par\noindent\rule{\textwidth}{0.4pt}\relax\vspace{1mm}
\noindent\rule{\textwidth}{0.4pt}\relax

\vspace{1cm}

\begin{center}
September, 1963
\end{center}

\vfill

\begin{center}
\large BRUSH ELECTRICAL ENGINEERING CO., LTD. \\
\small A MEMBER OF THE HAWKER SIDDELEY GROUP \\
\large LOUGHBOROUGH, ENGLAND \\
\end{center}


% ===== End Page ==================================================
\newpage

% use builtins
\ifTwoColumn
\begin{multicols}{2}
\small
\fi
\tableofcontents
\listoffigures
\listoftables

%\vfill

\begin{flushleft}
\textbf{Notation} \\
\vspace{1ex}
\begin{tabular}{llll}
$\xi$       & \multicolumn{3}{l}{any arithmetic expression} \\
$\xi_{v}$    & \multicolumn{3}{l}{any arithmetic expression with final mode floating point} \\
$\xi_{s}$    & \multicolumn{3}{l}{any arithmetic expression with final mode fixed point} \\
\mytt{ABC}  & floating point variables & ) & unless otherwise \\
\mytt{ILMN} & fixed point variables & ) & specified or implied \\
 & \multicolumn{3}{l}{Other notation explained in context} \\
\end{tabular}
\end{flushleft}
\ifTwoColumn
\end{multicols}
\fi

\vfill

% a horizontal rule
\par\noindent\rule{\textwidth}{0.4pt}\relax

% replicate the distribution list
\vspace{5mm}
\noindent
\begin{tabular}{p{0.5\textwidth}p{0.5\textwidth}}
\myuline{DISTRIBUTION} & \\
Dr. L.R. Blake & \\
Dr. D.A. Jones & Research Division Library \\
Central Reference Library & and as required \\
\end{tabular}


% ===== End Page ==================================================
\newpage

\setcounter{page}{1}
\cfoot{\thepage}

\begin{center}
\myuline{H-CODE MANUAL} \\
by D.C. Hogg \\
\end{center}


\ifTwoColumn
\small
\begin{multicols}{2}
\fi

\section{\myuline{A PROGRAMME}}

An H-Code programme (in the sense of a whole job)
consists of a number of \myuline{procedures} (or \myuline{subroutines})
which perform arithmetic and logic on fixed and
floating point variables represented by letters of the
alphabet.

There are two distinct parts of a programme (as
opposed to a procedure).

\renewcommand{\labelenumi}{(\roman{enumi})}
\begin{enumerate}
\def\theenumi{\roman{enumi}}

\item\label{it:BODY} \myuline{The body of a procedure} — that is the mnemonic
algebraic and logical instructions which are obeyed
dynamically performing the particular job for which
the programme has been composed.

\item\label{it:ORG} \myuline{The organisational instructions} which occur
outside the body of the procedures and are concerned
with the organisation and setting of the machine ready
to receive the procedures.

\end{enumerate}

Purely organisational instructions can occur inside
procedures but these are of a different type of those
under sect. (\ref{it:ORG}) being more concerned with the
particular procedure than with overall setting,

There are a different set of rules for (\ref{it:BODY}) and
(\ref{it:ORG}) and anything which is true for (\ref{it:BODY}) is not
necessarily true for (\ref{it:ORG}) and vice—versa,

\section{\myuline{"SETTING" THE MACHINE}}

In a statement such as "setting the machine" it
should be understood that it refers not to the machine
as such which is unchangeable in form, but to the
state of the store of the machine and more particularly
to the state of the H-Code translator which in normal
circumstances is permanently in (the store of) the
machine.  Once a programme is in the machine any
further procedures or data etc are understood to
be part of that programme as the machine, so to-speak,
is "conditioned" for that particular programme.
Therefore to get out of this state of affairs all
programmes should be headed by the instruction

\begin{addmargin}[1cm]{1em}%
\begin{lstlisting}
:SET
\end{lstlisting}
\end{addmargin}

which cancels all reference
to previous programmes held in the machine leaving the
whole of the available space free for the new programme.

The machine is now set up for the particular
programme it is to receive.  It has to be prepared
for the procedures of the programme, the variables
and integers.  These settings are introduced by

\begin{addmargin}[1cm]{1em}%
\begin{lstlisting}
:SETR :SETV :SETS
\end{lstlisting}
\end{addmargin}

In fact when a programme has one of
these three settings the \mytt{SET} is not needed as the
\myuline{first}

\begin{addmargin}[1cm]{1em}%
\begin{lstlisting}
:SETR :SETV :SETS
\end{lstlisting}
\end{addmargin}

implies

\begin{addmargin}[1cm]{1em}%
\begin{lstlisting}
:SET
\end{lstlisting}
\end{addmargin}

itself.  Therefore
in the great majority of programmes the instruction

\begin{addmargin}[1cm]{1em}%
\begin{lstlisting}
:SET
\end{lstlisting}
\end{addmargin}

need never be used.  Its practical use is limited
to introducing machine order programmes and special
programmes which can occur under

\begin{addmargin}[1cm]{1em}%
\begin{lstlisting}
:LIST
\end{lstlisting}
\end{addmargin}

\begin{addmargin}[1cm]{1em}%
\begin{lstlisting}
:SETR :SETV :SETS
\end{lstlisting}
\end{addmargin}

can occur in any order and more than
once but when the first "non-SET" organisational
instruction

\begin{addmargin}[1cm]{1em}%
\begin{lstlisting}
:DEFINE :LIST :LV :START :Z ...
\end{lstlisting}
\end{addmargin}

is encountered the setting instructions are understood
to have finished and the machine is finally set up
for that programme.  Any subsequent

\begin{addmargin}[1cm]{1em}%
\begin{lstlisting}
:SETR :SETV :SETS
\end{lstlisting}
\end{addmargin}

will cancel the programme.

\subsection{SETTING PROCEDURES}

Procedures are identified by words of \myuline{up to 4
letters}.  The programmer can choose any word with
which to identify a procedure as long as it does
not clash with any standard word which occurs in the
body of the procedure.  (see WORDS).  A word of 4
letters can have extra letters added to give it
meaning if necessary, but these extra letters have no
significance as far as the translator is concerned.
For example, we can \myuline{not} name two routines

\begin{addmargin}[1cm]{1em}%
\begin{lstlisting}
DESIGN A
DESIGN B
\end{lstlisting}
\end{addmargin}

\begin{flushleft}
as they are the same in the first 4 letters.
\end{flushleft}

In order that the machine is ready to input
procedures as they are given to it and so that a
procedure may refer to a procedure which is not yet
in the machine, we have to announce as part of the
setting instructions the names of all the procedures
that will go to make up the whole programme.

Reference numbers (q.v) are the basic framework on
which a procedure is built and referred to in
the machine.  In order that enough space be left for
this frame work each procedure name has, as a suffix
the \myuline{maximum} reference number which will occur in that
procedure.

A typical setting procedure would be:

\begin{addmargin}[1cm]{1em}%
\begin{lstlisting}
:SETR(ABC13,:XYZ(2),LMN)
\end{lstlisting}
\end{addmargin}


\renewcommand{\labelenumi}{\ifnum\value{enumi}=1 \myuline{Notes} (\roman{enumi})\else (\roman{enumi})\fi}
\begin{enumerate}

\item CR,LF,spsp\footnote{spsp = 2 consecutive spaces} are ignored before the words.

\item If there are no reference numbers no
suffix is needed.

\item Each word can be preceeded by \mytt{:} but this
is not necessary.

\end{enumerate}

\subsection{SETTING OF VARIABLES}

When a letter representing a variable (fixed or
floating point) is encountered in a procedure
reference is made to the corresponding entry in a
table which gives information on the variable
represented by that letter of the alphabet
(including special variable $\alpha$) e.g. whether the
variable is set, its mode (i.e. fixed - or floating
- point) and its absolute location (i.e. the storage
location in the machine which has been allocated to
$A_{0}$).

There are in fact two such tables:- that are
called the \myuline{local alphabet} and the \myuline{permanent alphabet}.
The local alphabet is the table which is referred to
in a procedure as explained above.  A copy is kept
of the initial settings in the permanent alphabet so
that after the local alphabet is altered by the use
of local variables (q.v.) the local alphabet can be
restored to its original state.  This and other uses
are explained in more detail below.


\subsection{SETV, SETS}

Any of the alphabet can be used to represent
a variable.  The programmer mast decide which letters
shall represent floating-point numbers and which shall
represent integer variables.  This does not restrict
to the use of 27 variables as a letter can have a
numerical suffix e.g. $A_{3}$, $N_{20}$, $\alpha _{10}$ giving us any
range of variables we need.

Before the procedures of a programme can be
input the machine must reserve space in the store
for the variables which are to be used.  This is done
using (organisational) instructions such as:

\begin{addmargin}[1cm]{1em}%
\begin{lstlisting}
:SETV(A23,N5,B)
:SETS(I10,J(3),K,L,M2)
\end{lstlisting}
\end{addmargin}

Those letters under \mytt{SETV} will represent floating—point
variables, those under \mytt{SETS} integer variables,
Letters can not occur under both \mytt{SETV} and \mytt{SETS}.  The
number is the highest suffix which the letter will need
throughout the programme.  If a letter occurs under
\mytt{SETV} or \mytt{SETS} more than once then the greatest suffix
is taken.

The word \mytt{SETV} or \mytt{SETS} can occur more than once
so that

\begin{addmargin}[1cm]{1em}%
\begin{lstlisting}
:SETV(A3,B10,C20)
:SETR(ABC3,XYZ)
:SETV(B10,A20,C30)
\end{lstlisting}
\end{addmargin}

Will be equivalent to:—

\begin{addmargin}[1cm]{1em}%
\begin{lstlisting}
:SETR(ABC3,XYZ)
:SETV(A20,B10,C30)
\end{lstlisting}
\end{addmargin}

\mytt{SET} (or the first \mytt{SETV}, \mytt{SETS} or \mytt{SETR} which
implies \mytt{SET}) automatically sets the orders

\begin{addmargin}[1cm]{1em}%
\begin{lstlisting}
:SETV(@9)
:SETS(O9)
\end{lstlisting}
\end{addmargin}

This means that $\alpha,\theta$ can only be SET as variable
and integer respectively. In fact $\alpha,\theta$ should not
(by convention) be used for any other purpose other
than in $\alpha$ — $\theta$ stores ( q.v. ) i.e. In transferring
data from one procedure to another.

When the first non-SET instruction is encountered
copies are made in both the local and permanent
alphabets of SET-tings as specified by
the \mytt{SETV} and \mytt{SETS} Instructions.


\begin{flushleft}
\myuline{Notes.}
\end{flushleft}

\renewcommand{\labelenumi}{(\roman{enumi})}
\begin{enumerate}

\item No check is made either in translation or in
running that the "SET" value of the suffix to a letter
has been exceeded, i.e. a variable has "gone too high".
In fact, it is true to say that 90\% of programme errors
not detected by the translator are due to this fault.

\item Throughout this manual "variables" will mean
floating-point variables and "integers" fixed—point
variables - where there is no ambiguity.

\item A can be used instead of $A_{0}$.

\item CR, LF spsp, are ignored before the letters on
the RHS of the \mytt{SETV}, \mytt{SETS} instructions.

\end{enumerate}

\section{\myuline{LOCAL VARIABLES}}

It is very convenient, for example, in writing
a procedure (Particularly of a mathematical or other
independent nature) which is to be used as a
subroutine (q.v.) by other procedures to be able to
use any letters to represent the variables and
integers in the subroutine without reference to
the meanings of the letters as \mytt{SET} in the master
programme.

To do this we use what are called \myuline{local variables}
(and \myuline{integers}).  When these variables are used
only the local alphabet is affected.  Setting of local
variables is done with orders such as

\begin{addmargin}[1cm]{1em}%
\begin{lstlisting}
:LV(A30,B10,X)
:LS(N2,L13)
:LR
\end{lstlisting}
\end{addmargin}

Those letters under \mytt{LV} will represent floating
point variables and those under \mytt{LS} integers.  When
this instruction is read the machine will, for
example, reserve 31 locations for $A_{0} \rightarrow A_{30}$ etc., and
make a note in the local alphabet of the new
definition of \mytt{A}.  As settings merely cancel any
previous setting of a particular letter then

\begin{addmargin}[1cm]{1em}%
\begin{lstlisting}
:LV(A30,A3)
\end{lstlisting}
\end{addmargin}

will reserve locations for $A_{30}$ then for $A_{3}$, the $A_{3}$,
superseding the $A_{30}$ The maximum suffix is \myuline{not} taken
as it is in \mytt{SETV}, \mytt{SETS}.

Inside \mytt{LV} or \mytt{LS} we can have a setting such as

\begin{addmargin}[1cm]{1em}%
\begin{lstlisting}
A →B30
\end{lstlisting}
\end{addmargin}

This means that \mytt{A} in the \myuline{local alphabet}, takes the
properties equivalent to \mytt{B30} in the \myuline{permanent alphabet}.
In effect this means that whenever \mytt{A} is met the
machine reads it as if it were $B_{30}$, $A_{1}$ as $B_{31}$ etc.

\st{Whether this transfer occurs under LV or LS is
irrelevant as the mode of the letter A for example,
takes its mode from B, The A can not have a suffix.}

\begin{framed}
\vspace{-1em}
\begin{flushleft}
\myuline{\textit{Hand written notes:}}
\end{flushleft}

\begin{addmargin}[1em]{1em}%
\begin{tabular}{lp{0.5cm}l}
\mytt{:L(A →B30)}  & &  Will have mode of \mytt{B30} \\
\mytt{:LS(A →B30)} & & Will be INTEGER regardless   \\
                   & & \hspace{2em} of mode of \mytt{B30} \\
\mytt{:LV(A →B30)} & & Will be FLOATING-POINT       \\
\end{tabular}
\end{addmargin}
\end{framed}

A setting such as

\begin{addmargin}[1cm]{1em}%
\begin{lstlisting}
A →A
\end{lstlisting}
\end{addmargin}

will replace the local meaning of \mytt{A} by the permanent
definition of \mytt{A}.

Inside a \mytt{LV} or \mytt{LS} we can also have a setting such
as

\begin{addmargin}[1cm]{1em}%
\begin{lstlisting}
A:3050
\end{lstlisting}
\end{addmargin}

This means that $A_{0}$ in the local alphabet is to be the
absolute location 3050 in the machine.  The mode is
determined by \mytt{LV} or \mytt{LS}.  This use of absolute locations
should only be used with great care as a knowledge of
the H-Code translator is needed.  It can be used to
overwrite parts of the translator which are not needed
by that particular programme in order to get more
working space.  No indication is made in this kind of
setting as to the maximum suffix of the letter.

The setting

\begin{addmargin}[1cm]{1em}%
\begin{lstlisting}
A:0
\end{lstlisting}
\end{addmargin}

irrespective of the mode of \mytt{A}, cancels the setting of
\mytt{A} in the local alphabet.  Thereafter \mytt{A}, for example,
will be rejected.  This means that a variable or integer
can not be set to absolute machine location 0.

These various settings can all occur under the same
\mytt{LV} or \mytt{LS} e.g.

\begin{addmargin}[1cm]{1em}%
\begin{lstlisting}
:LV(A30,B →C21,X:2000,W:0)
\end{lstlisting}
\end{addmargin}

Associated with the setting and manipulation of
local and permanent alphabets are the instructions:—

\begin{addmargin}[1cm]{1em}%
\begin{lstlisting}
:PL
:PR
\end{lstlisting}
\end{addmargin}

which transfers the permanent alphabet to the local
alphabet en masse, and

\begin{addmargin}[1cm]{1em}%
\begin{lstlisting}
:LP
\end{lstlisting}
\end{addmargin}

which transfers the local alphabet to the permanent
alphabet en masse.  Individual items can not be
transferred from the local alphabet to the permanent
alphabet.

A procedure or group of procedures which have their
own local variables should in general (although not
necessarily) be followed by \mytt{PL} so that the variables
then take on their permanent definition. e.g.

\begingroup
\centering
\makebox{%
\begin{picture}(70,60)
\thicklines

\put(10,52){\hbox{\kern3pt\mytt{:LV(A20,B10)}}}
\put(10,48){\hbox{\kern3pt\mytt{:LS(X,N2)}}}

\put(10,30){\polygon(0,0)(40,0)(40,15)(0,15)}
\put(15,37){\hbox{\kern3pt\small{Procedure}}}

\multiput(30,25)(0,2){3}{\line(0,1){1}}

\put(10,10){\polygon(0,0)(40,0)(40,15)(0,15)}
\put(15,17){\hbox{\kern3pt\small{Procedure}}}

\multiput(30,7)(0,2){2}{\line(0,1){1}}

\put(10,2){\hbox{\kern3pt\mytt{:PL}}}

\end{picture}}
\endgroup


\section{\myuline{LIST (DATA)}}

After reading the initial instruction \mytt{LIST} the
machine is in a position to accept.

\renewcommand{\labelenumi}{(\roman{enumi})}
\begin{enumerate}

\item Lists of variables in fixed-or floating-point
form or blocks of machine code introduced by the
address at which they are to be stored.

\item Blocks of H-Code programme which is translated,
obeyed "there and then", and then over written.

\end{enumerate}

\renewcommand{\labelenumi}{(\roman{enumi})}
\begin{enumerate}

\item It is very convenient to be able to set lists of
constants (which are unaltered throughout the running
of the programme) as the programme is translated
rather than as the programme is obeyed dynamically.  For
example:-

\noindent\begin{minipage}{\columnwidth}
\begin{lstlisting}
:LIST

A
+265 300.42
-22.6
C10=0 1 10
5 -6 2
@=3.6
N<
\end{lstlisting}
\hspace{1.5cm}\begin{math}
\Biggl\}{ \\
Machine-code \\
Block \\
}
\end{math}
\begin{lstlisting}
)
\end{lstlisting}
$\rightarrow 3000$
\begin{lstlisting}
<23A:04A2)
*
\end{lstlisting}
\end{minipage}

A number is not accepted unless the address is set.  The
address is the variable to which the first number is
to be sent.  As each number is read in the address is
updated by 1.  In the above example \mytt{A = 265, A1=300.42,
A2 = -22.6}.  If the variable is an integer, for example,
only integers can be listed under that address,
\mytt{< ... )} contains a machine code block which is stored
beginning at the current address. see \myuline{MACHINE CODE BLOCKS}.

Numbers and machine code orders can be mixed under
the same initial address, $\rightarrow 3000$ sets the absolute
location 3000 as the initial address.  Only integers or
machine code blocks can be under this address.

\item Under \mytt{LIST} between brackets \mytt{(} \hspace{2cm} \mytt{)} we can write
orders as if it were a procedure.  All
can be used other than reference numbers.
The order:

\begin{addmargin}[1cm]{1em}%
\begin{lstlisting}
:GOTO(26)
\end{lstlisting}
\end{addmargin}

would refer to absolute location 26.

\end{enumerate}


This facility is useful for setting up constant
data in translation which is too complicated to set as
constants, for example, to set up the series $1^{3}$,
$2^{3}$, ...... $100^{3}$ in $A_{1} \rightarrow A_{100}$ in translation we would
have:-

\begin{addmargin}[1cm]{1em}%
\begin{lstlisting}
:LIST
((0 →N)DO(100)
N:STAND →@*@*@ →AN
1ON(N)REPEAT)
*
\end{lstlisting}
\end{addmargin}

The above\footnote{should N start at 1 or just loop 101 times} would appear on the programme tape, and
the listing of $1^{3} \rightarrow 100^{3}$ would occur as part of the
translation process.  However, it is also very convenient
to be able to extract information from the machine once
a programme has run and, perhaps, got lost. This can be
done; For Example:

\begin{addmargin}[1cm]{1em}%
\begin{lstlisting}
:LIST('
'A:/'
'B:/'
N:1)
*
\end{lstlisting}
\end{addmargin}

Instead of using the \mytt{START} instruction e.g.

\begin{addmargin}[1cm]{1em}%
\begin{lstlisting}
:START(1,ABC)
\end{lstlisting}
\end{addmargin}

\begin{flushleft}
we could use
\end{flushleft}

\begin{addmargin}[1cm]{1em}%
\begin{lstlisting}
:LIST(:GOTO(1,ABC))
\end{lstlisting}
\end{addmargin}

when the programme would be obeyed at \mytt{(1,ABC)}
immediately.

In these "temporary" programmes the orders are
translated into the machine in the normal way and when
the closing bracket is encountered the orders are obeyed.
After the final order is obeyed control passes back
to the \mytt{LIST} input, and any further programme or local
variables will overwrite the "temporary" programme
hence no space is wasted.

A \mytt{LIST} is terminated by \hspace{1cm} \mytt{*} \hspace{1cm}, control
then passing back to the initial instruction input.

Various other characters occurring under LIST
effects of a trivial nature; these and the previous
facilities are summarised in the following table:- (Table \ref{tbl:LIST})

\multicolinterrupt{
\noindent\begin{minipage}{\textwidth}
\captionof{table}{Effect of characters occurring under \mytt{LIST}}\label{tbl:LIST}
\begin{tabular}{|l|l|}
\hline
$A\rightarrow Z$ \mytt{@} & with numerical suffix is address of variable or integer or machine orders. \\
\hline
$\rightarrow$ & with numerical suffix is address of integers or machine orders in absolute machine location. \\
\hline
\mytt{. + -} $0\rightarrow9$ & introduce integer or variable.  \myuline{Wrong} if address not set. \\
\hline
\mytt{<} & introduces machine code block.  \myuline{Wrong} if address not set. \\
\hline
\mytt{(\ \ \ )} & Contains H-Code orders to be obeyed “there and then". \\
\hline
\mytt{'\ \ \ '} & Contains characters which are to be copied directly onto output tape. \\
\hline
\mytt{)} & Computer STOPS. \\
\hline
\mytt{::} &  Computer WAITS.  Carries on when button 1 on Address 2 on the keyboard is changed. \\
\hline
\mytt{=} CR LF SpSp & are ignored. \\
\hline
\mytt{/ > ,} & are \myuline{Wrong}. \\
\hline
\mytt{*} & closes \mytt{LIST} \\
\hline
\end{tabular}
\end{minipage}}


\subsection{DATA}

The instruction \mytt{DATA} occurs in the body of a
procedure and is obeyed dynamically,  It has \myuline{exactly}
the same effect as \mytt{LIST} except that it reads off data
tape in running.  The final \mytt{*} will pass control back
to the running programme.

N.B. \mytt{LIST} can occur in the body of a procedure
(except for H-Code orders in ()) but this is to be
discouraged.  It is a relic of an early edition of the
translator.


\section{\myuline{THE BODY OF A PROCEDURE} \myuline{OBEYING A PROCEDURE}}

\subsection{THE BODY OF A PROCEDURE}

Procedures are introduced by an instruction such
as

\begin{addmargin}[1cm]{1em}%
\begin{lstlisting}
:DEFINE(ABC)
\end{lstlisting}
\end{addmargin}

\begin{flushleft}
where \mytt{ABC} is the name (which has been SET under \mytt{SETR})
\end{flushleft}

The procedure itself is terminated by the
instruction

\begin{addmargin}[1cm]{1em}%
\begin{lstlisting}
:CLOSE
\end{lstlisting}
\end{addmargin}

In translation control then passes back to that
part of the translator which deals with the
organisational instructions whence more procedures or
organisational instructions will be expected.

\subsection{OBEYING PROCEDURE}

The instruction, such as

\begin{addmargin}[1cm]{1em}%
\begin{lstlisting}
:START(3,ABC)
\end{lstlisting}
\end{addmargin}

\begin{flushleft}
\myuline{stores} the first order at which the programme is to be
obeyed.  When the computer is entered at 513 the
computer will start obeying the programme at this
point.
\end{flushleft}

\mytt{(3,ABC)} refers to REFERENCE 3 (q.v.) of procedure
\mytt{ABC}.  Note that we can only start obeying a programme
at a reference number.  The first order of a procedure
is reference 0 although it can not be labelled as such.
i.e. \mytt{(0,ABC)}

\begin{addmargin}[1cm]{1em}%
\begin{lstlisting}
:START(3000)
\end{lstlisting}
\end{addmargin}

or
\begin{addmargin}[1cm]{1em}%
\begin{lstlisting}
:START3000
\end{lstlisting}
\end{addmargin}

will start at absolute machine location 3000.

\begin{addmargin}[1cm]{1em}%
\begin{lstlisting}
:START(N3)
\end{lstlisting}
\end{addmargin}

will start at the order indicated by the value of $N_{3}$
when it is obeyed.  ($N_{3}$ is a SUFFIX (q.v.)).

Similarly

\begin{addmargin}[1cm]{1em}%
\begin{lstlisting}
:START(M,ABC)
\end{lstlisting}
\end{addmargin}

\begin{flushleft}
but this will be rarely used.
\end{flushleft}

The \mytt{START} instruction can occur anywhere in the
programme outside the procedure.

\section{\myuline{BINARY}}

This will cause a binary copy of the H-Code
programme in the computer to be output.

To input the binary tape the translator should be
in the machine and the normal entry to translate used.

If the 40 button on F1 is not depressed the
binary tape is compared on input with the programme
currently in the machine.  This serves as a check that
the programme has been correctly binarized.

As \mytt{BINARY} only outputs non-zero locations it is
advisable to use the instruction \mytt{Z} to clear the whole
of the programme store before translating a programme
which is to be binarized.

When the binary tape is re-input then the machine
is left in exactly the same state as when it was
output.  Hence new procedures and data can be input,
corrections can be made etc.

As the translator is not output by \mytt{BINARY} any
variables which overwrite parts of the translator with
such an order as

\begin{addmargin}[1cm]{1em}%
\begin{lstlisting}
:LS(N:200)
\end{lstlisting}
\end{addmargin}

can \myuline{not} be output by \mytt{BINARY}.  Therefore if these
variables are set by a \mytt{LIST} instruction then they will not
be binarized, therefore the \mytt{LIST} will have to be
input separately.

\section{\myuline{SUNDRY ORGANISATIONAL INSTRUCTIONS}}

\hspace{\parindent}\myuline{\mytt{STORE}}

The machine will print out (preceeded by LF) the
number of unused locations.  (terminated by \mytt{*}).

\myuline{\mytt{Z}}

All the locations in the machine, except those
occupied by the translator, will be cleared (i.e.
put equal to zero).

\myuline{\mytt{::}}

The computer will \myuline{wait} - until the last button on
the keyboard (1 on Address 2) is \myuline{changed} from its
current position.

\myuline{\mytt{'}\hspace{0.5cm}\mytt{'}}

All characters between the inverted commas will be
copied directly onto the output punch.


\section{\myuline{ERRORS IN PROCEDURE}}

If a coding error is detected by the translator
in the body of a procedure i.e. between the \mytt{DEFINE ( )}
and the final \mytt{CLOSE}, then that part of the procedure
already translated will be, in effect, erased
automatically and the corrected version of the
procedure can then be translated without any wastage
of space.

If an error in a procedure is discovered by the
programmer after the procedure has been fully translated,
he must take steps himself to erase the previous
version of the procedure before attempting to
translate the new one. This he does with an order such
as

\begin{addmargin}[1cm]{1em}%
\begin{lstlisting}
:X(ABC)
\end{lstlisting}
\end{addmargin}

which will erase reference to the previous version of
\mytt{ABC}. Then follows the new version of \mytt{ABC} beginning with,
as normal

\begin{addmargin}[1cm]{1em}%
\begin{lstlisting}
:DEFINE(ABC)
\end{lstlisting}
\end{addmargin}

The \mytt{X} cancellation does not allow re-use of the
space used by the original version of \mytt{ABC} therefore
the "re-defining" of procedures is conditional on there
being sufficient store left in the machine.

\section{\myuline{SUMMARY OF INITIAL (OR ORGANISATIONAL) INSTRUCTIONS}}

See: Table \ref{tbl:ORG}


\multicolinterrupt{
\noindent\begin{minipage}{\textwidth}
\captionof{table}{Summary of initial (organisational) instructions}\label{tbl:ORG}
\begin{tabular}{|l|l|l|}
\hline
Group A & \mytt{SET}       & Sets up machine to receive new programme \\
\hline
Group B & \mytt{SETV( , )} & Sets variables in permanent and local alphabets \\
        & \mytt{SETS( , )} & Sets integers in permanent and local alphabets \\
        & \mytt{SETR( , )} & Sets procedures with given names and given references \\
\hline
Group C & \mytt{DEFINE( )} & Introduces the body of a procedure \\
        & \mytt{LIST}      & Lists of variables and blocks of H-Code \\
        & \mytt{LV( , )}   & variables and integers respectively:- \\
        & \mytt{LS( , )}   & $A$, sets variable in local alphabet \\
        &                  & $A\rightarrow B_{20}$, $A$ in local alphabet takes properties of $B_{20}$ in permanent \\
        &                  & $A:3000$, $A$ in local alphabet becomes absolute location 3000 \\
        & \mytt{PL}        & Permanent alphabet to local alphabet \\
        & \mytt{LP}        & Local alphabet to permanent alphabet \\
        & \mytt{STORE}     & Prints store left \\
        & \mytt{START( )}  & Stores reference and procedure for starting at 513 \\
        & \mytt{BINARY}    & Binarizes the programme in store \\
\hline
Group D & \mytt{X( )}      & Cancels the given procedure \\
        & \mytt{Z}         & Clears that part of store used by programme and data \\
\hline
Group E & \mytt{::}        & Wait \\
        & \mytt{'}\hspace{0.5cm}\mytt{'} & Title between is copied \\
\hline
\end{tabular}

\vspace{0.2cm}
\myuline{Notes} \\
The first word of Group B which occurs before a word of Group C has also the effect of Group A i.e. \mytt{SET}.
\end{minipage}}


\section{\myuline{FLOATING POINT FORM OF NUMBERS}}

\subsection{FLOATING POINT FORM}

A common way of writing very large or very small
numbers in scientific work is to associate a power
of 10 with each of them; for example:

\begin{tabular}{lrl}
   & - & $35,742,000,000,000,000$ may be written: \\
   & - & $3.5742 \times 10^{16}$ \\
or &   & $0.000,000,000,000,123,4$ may be written: \\
   &   & $.1234 \times 10^{-12}$ \\
\end{tabular}

That is to say the numerical part, or \myuline{argument}, or
mantissa is written with the decimal point in a
convenient position and the \myuline{exponent} of 10 is an integer
adjusted to make the overall value of the number
correct.  This way of representing numbers is generally
called \myuline{floating-point form}, as opposed to \myuline{fixed-point
form} of numbers written without an exponent.  Thus
within a computer a number can be represented in this
form by 2 quantities, the argument (a fraction) and
the exponent (an integer) \myuline{packed} into one \uline{computer
word} — or \myuline{zero}.

When a number is printed out by methods (\ref{it:PRMS}) and
(\ref{it:PRS}) in PRINT STATEMENT it is printed out in what is
called \myuline{Standardised Floating-point form}.  In this if
$A10^{b}$ is the number then A and b are adjusted so that

if $A \neq 0$ then $.1 \leqslant A < 1$ when b is positive or
negative integer.

if $A = 0$ then $A = 0$ and $b = -78$ (b is irrelevant
in fact).
\vspace{2pt}

Users need not be conscious of the form of
floating-point except when they are printing variables
in floating point form — and to a lesser extent when
preparing data in this form.


\subsection{FORM OF NUMBERS}

\renewcommand{\labelenumi}{(\alph{enumi})}
\begin{enumerate}
\def\theenumi{\alph{enumi}}

\item\label{it:IAO} \myuline{In Arithmetic Orders}

In arithmetical statements a variable can be
replaced by an \myuline{unsigned} constant (except, of course
that we cannot write $\rightarrow 23$)

\myuline{Integers} must not have decimal points, and they must
be less than $2^{38}$ i.e. 274,877,906,944

\myuline{Floating point} numbers should, in general, have a
decimal point, however if the mode of the part of the
expression has been determined the decimal point may
be omitted. The integral part of the number must be less
than $2^{38}$ and the 'integer value' of the fractional part
(i.e. the fraction treated as an integer e.g. .0001234
would be 1234) also less than $2^{38}$.  An exponent is not
allowed.

A number in arithmetic is terminated by the first
non-numeric character.

\item \myuline{In data for \mytt{READ}, \mytt{LIST}, \mytt{DATA}}

In data a negative number must be signed, a
positive sign is optional.

\myuline{Integers} as in (\ref{it:IAO})

\myuline{Floating point numbers} as in (\ref{it:IAO}), but decimal point is
not necessary in integral numbers.  A decimal exponent
is allowed — the argument is followed by a / followed
by the decimal exponent which is a signed or unsigned
integer $\leqslant 77$ in absolute magnitude (the maximum number
allowed in the machine is approximately ($\pm 1.16 \times 10^{77}$)

e.g. -13/-2 or 2.63/7

A number in data must be terminated by CR, spsp or
LF.

\item \myuline{In machine code blocks}

In machine code the number \myuline{must} be signed.

\myuline{Integers} as in (\ref{it:IAO})

\myuline{Floating point} numbers as in (\ref{it:IAO}), but the decimal
point must be used.  An exponent is \myuline{not} allowed.

A number in machine code must be terminated by LF
or, in the case of integers, by \mytt{<}
\end{enumerate}


\section{\myuline{MACHINE CODE BLOCKS}}

Blocks of machine code orders occur in the body
of a procedure amongst the ordinary mnemonic statements
where they are translated and obeyed in sequence with
the other statements.

They also occur under \mytt{LIST} outside (or inside) a
procedure where they are stored in the address as
specified by \mytt{LIST}.

Although the full T102 (Elliott machine code input.
routine) facilities are not available enough are
provided for the vast majority of uses of machine code.
The main difference is that \myuline{fixed point fractions are
not allowed}.  We are restricted to integer and floating-point working.

A block of machine code is introduced by \mytt{<}
and closed by \mytt{)}

In the block we are allowed the following types
of words:—

\noindent\myuline{Normal orders}

\begin{tabular}{l}
$p_{1}q_{1}N_{1}/p_{2}q_{2}N_{2}$ \\
$/p_{2}q_{2}N_{2}$ \\
$p_{1}q_{1}N_{1}/$ \\
$/$ \\
\end{tabular}

where \mytt{/} can be replaced by \mytt{:}.

p and q are the integers 0 — 7
and N is of the forms

\begin{tabular}{ll}
\mytt{123}   & i.e. (integer) \\
\mytt{123,}  & \mytt{,} is relative to first order of \\
\mytt{-123,} & block \\
\mytt{-123}  & \\
\mytt{A12} or \mytt{@12} & \\
\mytt{(3)} or \mytt{(3,ABC)} & i.e. the jump suffix (use with \\
               & 40 — 43 instructions)
\end{tabular}

\noindent\myuline{Octal}

8 followed by the 13 octal (0-7) elements of the
word.

\noindent\myuline{Numbers}

Introduced by $\pm$.
The presence, and only the presence, of a decimal
point will make the number into a floating point
number - otherwise it will be an integer.

In floating point number \mytt{/} followed by decimal
exponent is not allowed.

Fixed point fractions are not allowed.
An order such as

\begin{addmargin}[1cm]{1em}%
\begin{lstlisting}
A12
\end{lstlisting}
\end{addmargin}

will represent the address of $A_{12}$ as an integer.

- An integer followed by \mytt{,} will give the integer
plus the address of 1st order of block

\begin{tabular}{ll}
e.g. & \mytt{+13,} \\
     & \mytt{~1,} \\
\end{tabular}

\noindent\myuline{New address} (Allowed \myuline{only under \mytt{LIST}})

An integer followed by \mytt{<} will alter the address
for subsequent, orders — but \mytt{,} will \myuline{still be relative
to the beginning of the block.} e.g.

\begin{tabular}{ll}
\mytt{A12<} \\
\mytt{+13<} \\
\mytt{-1,<} \\
\end{tabular}

\noindent\myuline{An order must be closed} by LF or \mytt{)}

\noindent\myuline{Systems of Blocks} of machine code are not the same as
T102.  Each block is 'put into a letter' and reference
made from one block to another with an order such as

\begin{addmargin}[1cm]{1em}%
\begin{lstlisting}
73A:40A1
\end{lstlisting}
\end{addmargin}

which would use the block in \mytt{A} as a subroutine. An
order such as

\begin{addmargin}[1cm]{1em}%
\begin{lstlisting}
730,2:401,2
\end{lstlisting}
\end{addmargin}

\begin{flushleft}
has no meaning as Block 2 is not defined.
\end{flushleft}

This system will not effect the use of library
routines as blocks of programme as they generally only
refer to themselves and not to other blocks.  The only
stumbling block to full use of Elliott machine code
library routines is probably the lack of fixed point
fractions and these appear to be rarely used - if at
all.

\section{\myuline{USE OF PROCEDURES}}

A programme is made up, in general, of a group of
procedures each of which does its own job using the other
procedures as subroutines or passing control to them in
turn.

\vspace{1ex}
\noindent\myuline{By direct entry}

In this method control is passed from the current
procedure to another by 'jumping into' it with an order
such as

\begin{addmargin}[1cm]{1em}%
\begin{lstlisting}
:IF(X)ZERO:GOTO(3,ABC)
\end{lstlisting}
\end{addmargin}

Here control passes to beyond reference 3 of
procedure \mytt{:ABC}.  No \myuline{link} is set - i.e. no marker is
kept of the position or the procedure from which control
was passed.

\vspace{1ex}
\noindent\myuline{As a Subroutine}

In this method control is passed to the \myuline{Subroutine}
but a \myuline{link} is set and when the statement

\begin{addmargin}[1cm]{1em}%
\begin{lstlisting}
:EXIT
\end{lstlisting}
\end{addmargin}

occurs in the subroutine control is passed back to the
order after the statement which calls on the subroutine,
which is typified by

\begin{addmargin}[1cm]{1em}%
\begin{lstlisting}
:ABC(N3+1)
\end{lstlisting}
\end{addmargin}

i.e. the name of the subroutine procedure together with
a \myuline{simple suffix} referring to the reference number at
which entry is to be made. Diagrammatically:—

\vspace{0.5cm}
\begingroup
\centering
\fbox{%
\begin{picture}(70,60)
\thicklines

\put(3,55){\hbox{\kern3pt\mytt{Master Program}}}
\put(40,55){\hbox{\kern3pt\mytt{Subroutine XYZ}}}

\put(5,7){\line(0,1){40}}
\put(30,7){\line(0,1){40}}
\qbezier(5,7)(11,1)(17,7)
\qbezier(17,7)(23,13)(30,7)
\qbezier(5,47)(11,41)(17,47)
\qbezier(17,47)(23,53)(30,47)

\put(40,7){\line(0,1){40}}
\put(65,7){\line(0,1){40}}
\qbezier(40,7)(46,1)(52,7)
\qbezier(52,7)(58,13)(65,7)
\qbezier(40,47)(46,41)(52,47)
\qbezier(52,47)(58,53)(65,47)


\put(5,26){\hbox{\kern3pt\mytt{:XYZ3}}}

\put(40,35){\hbox{\kern3pt\mytt{3)}}}
\put(40,15){\hbox{\kern3pt\mytt{:EXIT}}}

% MP
\put(10,43){\vector(0,-1){15}}
\put(10,25){\vector(0,-1){15}}

% Call
\Vector(15,28)(40,37)
\Vector(41,16)(14,25)

% XYZ
\put(50,35){\vector(0,-1){18}}

\end{picture}}

\captionof{figure}{The use of a subroutine}\label{fig:scall}
\endgroup
\vspace{0.5cm}


As there is only one link associated with each
procedure a procedure can not be entered as a subroutine
from itself or from any sub-sub..-routine etc., of that
procedure.  Although there is no objection: to using a
part of the same procedure as a subroutine if exit from
the 'master part' of the procedure is not by obeying
link (i.e. by \mytt{EXIT}). (i.e. no \myuline{recursive} use of
procedures).  Apart from the above restriction there
is no limit to the depth of subroutine.  Although in
practice it is, of course, limited by the number of
procedures which make-up the programme as a whole.


\section{\myuline{STRUCTURE OF A PROCEDURE}}

The basic structure of a procedure (\mytt{ABC} say) is

\begin{addmargin}[1cm]{1em}%
\begin{lstlisting}
:DEFINE(ABC) , , , , , , :CLOSE
\end{lstlisting}
\end{addmargin}

Between the commas are statements — such as

\begin{tabular}{l}
Arithmetical statements \\
Conditional statements \\
Machine Code Blocks \\
Read Statements \\
Defined Functions \\
etc \\
etc \\
\end{tabular}

However, many statements themselves imply comma,
so that we can string such statements together without
actually using a comma.  Similarly line-feed implies
comma hence statements can be put on separate lines without commas.

The omitting of commas can make the print-out look
tidier, but care should be taken to ensure that when a
word implies a comma and the new statement begins with
a letter then there should be some non-alphabet
character between the two statements — e.g. comma itself
or spsp. e.g. we can write

\begin{addmargin}[1cm]{1em}%
\begin{lstlisting}
:IF(A=B)THEN 1.0 →C:,
\end{lstlisting}
\end{addmargin}

but we \myuline{cannot} write

\begin{addmargin}[1cm]{1em}%
\begin{lstlisting}
:IF(A=B)THEN X →C:,
\end{lstlisting}
\end{addmargin}

as the \mytt{X} would be taken to be part of \mytt{THEN} (see
WORDS).  We can get over this in many ways.  e.g.

\begin{addmargin}[1cm]{1em}%
\begin{lstlisting}
:IF(A=B)THEN,X →C:,
:IF(A=B)THEN(X →C):,
:IF(A=B)THEN  X →C:,
\end{lstlisting}
\end{addmargin}

or

\begin{addmargin}[1cm]{1em}%
\begin{lstlisting}
:IF(A=B)THEN
X →C:,
\end{lstlisting}
\end{addmargin}

Details of those statements which imply comma are
given in the summary of facilities.

\section{\myuline{LOOP AND DO INSTRUCTIONS}}

\newcommand{\myloopcount}{$\xi_{s}$}

The Instructions:—

\renewcommand{\labelenumi}{(\roman{enumi})}
\begin{enumerate}
\def\theenumi{\roman{enumi}}

\item\label{it:LOOPU}\ \\ \vspace{-5pt}

\begin{addmargin}[1cm]{1em}%
\makebox{%
\begin{picture}(50,14)
\thicklines

\put(3,12){\hbox{\kern3pt\mytt{:LOOP}}}
\put(3,1){\hbox{\kern3pt\mytt{:REPEAT}}}

\put(1,11){\vector(1,0){3}}
\put(1,11){\line(0,-1){9}}
\put(1,2){\line(1,0){3}}

\thinlines
\put(8,5){\line(0,1){1}}
\put(8,7){\line(0,1){1}}
\put(8,9){\line(0,1){1}}

\end{picture}}
\end{addmargin}

which must always occur as a pair facilitate the
programming of repetitive processes.  The programme
between the \mytt{LOOP} and \mytt{REPEAT} is obeyed repetitively,
the \mytt{REPEAT} instruction sending control back to the
instruction immediately following the \mytt{LOOP} instruction.
\mytt{LOOP} has no effect other than introducing a loop.

\item\label{it:LOOPCOND}\ \\ \vspace{-5pt}

\begin{addmargin}[1cm]{1em}%
\makebox{%
\begin{picture}(50,14)
\thicklines

\put(3,12){\hbox{\kern3pt\mytt{:LOOP}}}
\put(3,1){\hbox{\kern3pt\mytt{:IF - - - - - - :REPEAT}}}

\put(1,11){\vector(1,0){3}}
\put(1,11){\line(0,-1){9}}
\put(1,2){\line(1,0){3}}

\thinlines
\put(8,5){\line(0,1){1}}
\put(8,7){\line(0,1){1}}
\put(8,9){\line(0,1){1}}

\end{picture}}
\end{addmargin}

This is similar to (\ref{it:LOOPU}) except that the \mytt{REPEAT} is
conditional.  The \mytt{IF} can introduce any kind or
complexity of conditional statement.

\item\label{it:DOLOOP}\ \\ \vspace{-5pt}

\begin{addmargin}[1cm]{1em}%
\makebox{%
\begin{picture}(50,14)
\thicklines

\put(3,12){\hbox{\kern3pt\mytt{:DO(}$\xi_{s}$\mytt{)}}}
\put(3,1){\hbox{\kern3pt\mytt{:REPEAT}}}

\put(1,11){\vector(1,0){3}}
\put(1,11){\line(0,-1){9}}
\put(1,2){\line(1,0){3}}

\thinlines
\put(8,5){\line(0,1){1}}
\put(8,7){\line(0,1){1}}
\put(8,9){\line(0,1){1}}

\end{picture}}
\end{addmargin}

The programme between the \mytt{DO} and the \mytt{REPEAT} is
obeyed \myloopcount, times — i.e. the value of \myloopcount, when the
programme is obeyed, The \mytt{REPEAT} associated with a \mytt{DO}
must be unconditional.

\mytt{LOOP} loops and \mytt{DO} loops can occur 'inside' other
loops to a depth of 9. A \mytt{REPEAT} is paired with the last
impaired \mytt{LOOP} or \mytt{DO}. e.g.

\begingroup
\centering
\makebox{%
\begin{picture}(70,50)
\thicklines

\put(15,46){\hbox{\kern3pt\mytt{:DO(...)}}}
\put(15,38){\hbox{\kern3pt\mytt{:LOOP}}}
\put(15,35){\hbox{\kern3pt\mytt{:LOOP}}}
\put(15,26){\hbox{\kern3pt\mytt{:REPEAT}}}
\put(15,23){\hbox{\kern3pt\mytt{:DO(...)}}}
\put(15,13){\hbox{\kern3pt\mytt{:REPEAT}}}
\put(15,10){\hbox{\kern3pt\mytt{:IF (...) REPEAT}}}
\put(15,1){\hbox{\kern3pt\mytt{:REPEAT}}}

\put(6,46){\vector(1,0){10}}
\put(6,46){\line(0,-1){44}}
\put(6,2){\line(1,0){10}}

\put(8,38){\vector(1,0){8}}
\put(8,38){\line(0,-1){27}}
\put(8,11){\line(1,0){8}}

\put(10,35){\vector(1,0){6}}
\put(10,35){\line(0,-1){8}}
\put(10,27){\line(1,0){6}}

\put(10,23){\vector(1,0){6}}
\put(10,23){\line(0,-1){9}}
\put(10,14){\line(1,0){6}}

\end{picture}}
\endgroup


Other programme can occur anywhere between the
\mytt{LOOP}s \mytt{DO}s and \mytt{REPEAT}s.

\end{enumerate}

\renewcommand{\labelenumi}{\ifnum\value{enumi}=1 \myuline{Note} (\roman{enumi})\else (\roman{enumi})\fi}
\begin{enumerate}

\item If \myloopcount, in \mytt{DO(}\myloopcount\mytt{)} is negative then there will
be error signal.

If \myloopcount  is zero then it will be equivalent to \mytt{DO(1)}\footnote{handwritten note: \mytt{"ERROR"} here}.
However if we want literally to \mytt{DO} a loop zero times
i.e. by-pass the loop we would write something like:

\begin{addmargin}[1cm]{1em}%
\makebox{%
\begin{picture}(50,14)
\thicklines

\put(3,12){\hbox{\kern3pt\mytt{:IF(N)NOT ZERO:THEN:DO(N)}}}
\put(3,1){\hbox{\kern3pt\mytt{:REPEAT:,}}}

\put(1,11){\vector(1,0){3}}
\put(1,11){\line(0,-1){9}}
\put(1,2){\line(1,0){3}}

\thinlines
\put(8,5){\line(0,1){1}}
\put(8,7){\line(0,1){1}}
\put(8,9){\line(0,1){1}}

\end{picture}}
\end{addmargin}

\item \myuline{Example}

Calculate

\begin{addmargin}[1cm]{1em}%
\begin{math}
C = \displaystyle\sum_{I=0}^{9}{A_{I}B_{2I}}
\end{math}
\end{addmargin}

This can be done in various ways:

\renewcommand{\labelenumii}{\ifnum\value{enumii}=1 e.g. (\alph{enumii})\else or (\alph{enumii})\fi}
\begin{enumerate}

\item
\begin{addmargin}[1cm]{1em}%
\begin{lstlisting}
0.0 →C
(0 →I →I1)DO(10)
(AI+BI1)ON(C)
1ON(I,I1,I1)REPEAT
\end{lstlisting}
\end{addmargin}

\item
\begin{addmargin}[1cm]{1em}%
\begin{lstlisting}
0.0 →C
(0 →I)DO(10)
(AI+B'(I+I))+C →C
1ON(I)REPEAT
\end{lstlisting}
\end{addmargin}

\end{enumerate}
\end{enumerate}


\section{\myuline{ON STATEMENT}}

The statement:—

\begin{addmargin}[1cm]{1em}%
\vspace{0.2cm}
$\xi$:\mytt{ON(P,Q,R,...)}
\end{addmargin}

where:

$\xi$ is fixed or floating point expression

$\xi$, \mytt{P}, \mytt{Q}, \mytt{R...} are all same mode where \mytt{P}, \mytt{Q}, \mytt{R...}
are any number and can have simple or complex suffixes.

Is such that the function \mytt{ON} adds the value of the
accumulator onto each of the variables specified on
the right hand side.

\begin{flushleft}
e.g.
\end{flushleft}

\begin{addmargin}[1cm]{1em}%
\begin{lstlisting}
(N+M)ON(A3,B'(P+Q))
\end{lstlisting}
\end{addmargin}

\begin{flushleft}
will perform the operations
\end{flushleft}

\begin{addmargin}[1cm]{1em}%
\begin{lstlisting}
N+M+A3 →A3
N+M+B'(P+Q) →B'(P+Q)
\end{lstlisting}
\end{addmargin}

\begin{flushleft}
and
\end{flushleft}

\begin{addmargin}[1cm]{1em}%
\begin{lstlisting}
1:ON(L,M,N)
\end{lstlisting}
\end{addmargin}

\begin{flushleft}
is equivalent to
\end{flushleft}

\begin{addmargin}[1cm]{1em}%
\begin{lstlisting}
L+1 →L
M+1 →M
N+1 →N
\end{lstlisting}
\end{addmargin}


\begin{flushleft}
\myuline{Notes}
\end{flushleft}

\renewcommand{\labelenumi}{(\roman{enumi})}
\begin{enumerate}

\item With integers \mytt{ON} is always more efficient than the
appropriate arithmetic.  But with floating point
variables its use, as far as efficiency is concerned,
becomes debatable except when $\xi$ is complicated or if
any of the RHS have complex suffices.

\item \mytt{ON} implies comma except that after final \mytt{)} a
letter-introduces a word.  This is to allow the
statement such as

\begin{addmargin}[1cm]{1em}%
\begin{lstlisting}
1:ON(N)REPEAT
\end{lstlisting}
\end{addmargin}

which is the most common use of \mytt{ON}.

\item After \mytt{,} or \mytt{(} CR,LF, spsp are ignored.

\end{enumerate}


\section{\myuline{SUFFICES}}

Integers or integer expressions can be used as suffices
to floating- or fixed-point variables.

\subsection{SIMPLE SUFFICES}

The following examples cover all the allowable
types of SIMPLE SUFFICES.

\begin{addmargin}[1cm]{1em}%
\begin{tabular}{lrl}
\mytt{A13}  & or & \mytt{A(13)}  \\
\mytt{AN}   & or & \mytt{A(N)}   \\
\mytt{AN15} & or & \mytt{A(N15)} \\
\multicolumn{2}{r}{\mytt{A(N+20)}} & \\
\multicolumn{2}{r}{\mytt{A(N5-6)}} & \\
\end{tabular}
\end{addmargin}

Simple suffices can be written in \myuline{this} and \myuline{only this}
form. e.g.

\begin{addmargin}[1cm]{1em}%
\begin{math}
A_{(n_{1}+20)}
\end{math}
\end{addmargin}

\begin{flushleft}
must be written as:
\end{flushleft}

\begin{addmargin}[1cm]{1em}%
\begin{lstlisting}
A(N1+20)
\end{lstlisting}
\end{addmargin}

\begin{flushleft}
\mytt{A(20+N1)} will not be accepted.
\end{flushleft}

\begin{addmargin}[1cm]{1em}%
\begin{math}
A_{N_{15}}
\end{math}
\end{addmargin}

\begin{flushleft}
must be written as:
\end{flushleft}

\begin{addmargin}[1cm]{1em}%
\begin{lstlisting}
AN15   or   A(N15)
\end{lstlisting}
\end{addmargin}

\begin{flushleft}
\mytt{A(N(15))} will not be accepted.
\end{flushleft}

\subsection{COMPLEX SUFFICES}

These are written

\begin{addmargin}[1cm]{1em}%
\begin{flushleft}
\mytt{A'(}$\xi_{s}$\mytt{)}
\end{flushleft}
\end{addmargin}

$\xi_{s}$ itself can contain integers or variables which
have simple suffices themselves, \myuline{but not complex suffices}

\begin{flushleft}
Examples:
\end{flushleft}

\begin{addmargin}[1cm]{1em}%
\begin{lstlisting}
A'(P1+Q2)
A'(B:INT)
A'(NM3+N(M4+1)-5)
I'((N+M)*(N-M))
I'((N+M)SQ)
\end{lstlisting}
\end{addmargin}

\begin{flushleft}
\myuline{Notes}
\end{flushleft}

\renewcommand{\labelenumi}{(\roman{enumi})}
\begin{enumerate}

\item It is important that the value of a suffix is $\geqslant 0$
and $\leqslant 8191$ as values outside this range cause the arithmetic order to be destroyed.

\item A suffix should be represented as a simple suffix
where possible as complex suffices are less efficient
in time and space.

\item Complex suffices can be used in all types of
arithmetic except division by an integer.  For details
see Arithmetic.

\item As teleprinters have no multi-level printing
facilities suffices are printed on the same level as
the variable which they qualify but it should be
remembered that \mytt{A4} means $A_{4}$

\end{enumerate}


\section{\myuline{JUMP SUFFICES,} \myuline{UNCONDITIONAL JUMPS}}

\subsection{JUMP SUFFICES}

The jump suffix \mytt{(\rlap{/}S)} can be of the following forms

\renewcommand{\labelenumi}{(\roman{enumi})}
\begin{enumerate}
\def\theenumi{\roman{enumi}}
\item\label{it:JMPSM} as an ordinary SIMPLE SUFFIX
\item\label{it:JMPCX} \begin{addmargin}[1cm]{1em}%
\begin{lstlisting}
(3,ABC)
(N4+5,ABC)
\end{lstlisting}
\end{addmargin}
\end{enumerate}

Jump suffices are used to qualify the \mytt{START} and
\mytt{GOTO} statements.

Type (\ref{it:JMPSM}) will refer to reference numbers within the
current procedure, or if used outside a procedure, to
the absolute machine location.

Type (\ref{it:JMPCX}) will refer to reference numbers in the
specified procedure.

\subsection{UNCONDITIONAL JUMPS}

The statement
\begin{addmargin}[1cm]{1em}%
\begin{flushleft}
\mytt{:GOTO(\rlap{/}S)}
\end{flushleft}
\end{addmargin}

\begin{flushleft}
will cause control to pass to the order immediately
following the reference number specified by \mytt{(\rlap{/}S)} (see
JUMP SUFFIX above).
\end{flushleft}

\begin{flushleft}
e.g.
\end{flushleft}

\begin{addmargin}[1cm]{1em}%
\begin{lstlisting}
:GOTO3
:GOTO(N,ABC)
:GOTO(N)
\end{lstlisting}
\end{addmargin}

\begin{flushleft}
(note that: \mytt{GOTO N} would be wrong as \mytt{N} would be taken
as part of the word \mytt{GOTO}).
\end{flushleft}


\section{\myuline{CONDITIONAL STATEMENTS}}

There are two methods of writing conditions, one
of which is more efficient than the other (in so far
as it uses less space and is dynamically faster).  The
less efficient method is used because it is rather
easier to use when a programme is being written
quickly - say for a one-off job, or when mathematical
clarity is needed.

\renewcommand{\labelenumi}{\myuline{Type \theenumi}}
\begin{enumerate}

\item\label{it:EFF} (Efficient)

\begin{tabular}{ll}
($\xi$)\mytt{NEGATIVE}     & \mytt{<0} \\
($\xi$)\mytt{ZERO}         & \mytt{=0} \\
($\xi$)\mytt{POSITIVE}     & \mytt{>0} \\
($\xi$)\mytt{NOT NEGATIVE} & \mytt{=0 >0} \\
($\xi$)\mytt{NOT ZERO}     & \mytt{<0 >0} \\
($\xi$)\mytt{NOT POSITIVE} & \mytt{<0 =0} \\
\end{tabular}

\item\label{it:LEF} (Less efficient than type \ref{it:EFF})

\begin{tabular}{ll}
($\xi$ \mytt{<}  $\xi'$) & \\
($\xi$ \mytt{=}  $\xi'$) & \\
($\xi$ \mytt{>}  $\xi'$) & \\
($\xi$ \mytt{:<}  $\xi'$) & \mytt{= or >} \\
($\xi$ \mytt{:=}  $\xi'$) & \mytt{< or >} \\
($\xi$ \mytt{:>}  $\xi'$) & \mytt{< or =} \\
\end{tabular}

\end{enumerate}

\begin{flushleft}
where $\xi$ and $\xi'$ are the same modes

$\nless$ is punched \mytt{:<}, $\ne$ \mytt{:=}, $\ngtr$ \mytt{:>}
\end{flushleft}

\subsection{GENERAL CONDITIONAL STATEMENT}

If we write \mytt{(\rlap{/}C)} for the general condition (any of
above) the general conditional statement is:-

\begingroup
\centering
\makebox{%
\begin{picture}(70,25)
\thicklines

\put(15,17){\hbox{\kern3pt\mytt{:IF(\rlap{/}C)\ AND\ \ (\rlap{/}C') }}}
\put(15,13){\hbox{\kern3pt\mytt{\ \ \ \ \ \ \ \ OR}}}
\put(15,9){\hbox{\kern3pt\mytt{\ \ \ \ \ \ \ \ REPEAT}}}
\put(15,5){\hbox{\kern3pt\mytt{\ \ \ \ \ \ \ \ GOTO(\rlap{/}S)}}}
\put(15,1){\hbox{\kern3pt\mytt{\ \ \ \ \ \ \ \ THEN}}}

\ifTwoColumn
\put(28,22){\vector(0,-1){3}}
\put(28,22){\line(1,0){13}}
\put(41,22){\line(0,-1){4}}
\put(41,18){\line(-1,0){2}}

\put(25,19){\line(1,1){1}}
\put(25,1){\line(0,1){8}}
\put(25,11){\line(0,1){8}}
\put(24,10){\line(1,1){1}}
\put(24,10){\line(1,-1){1}}
\put(25,1){\line(1,-1){1}}

\put(32,13){\line(-1,-1){1}}
\put(32,13){\line(0,1){2}}
\put(32,15){\line(1,1){1}}
\put(32,17){\line(1,-1){1}}
\put(32,17){\line(0,1){2}}
\put(32,19){\line(-1,1){1}}

\else

\put(31,22){\vector(0,-1){3}}
\put(31,22){\line(1,0){16}}
\put(47,22){\line(0,-1){4}}
\put(47,18){\line(-1,0){2}}

\put(27,19){\line(1,1){1}}
\put(27,1){\line(0,1){8}}
\put(27,11){\line(0,1){8}}
\put(26,10){\line(1,1){1}}
\put(26,10){\line(1,-1){1}}
\put(27,1){\line(1,-1){1}}

\put(35,13){\line(-1,-1){1}}
\put(35,13){\line(0,1){2}}
\put(35,15){\line(1,1){1}}
\put(35,17){\line(1,-1){1}}
\put(35,17){\line(0,1){2}}
\put(35,19){\line(-1,1){1}}

\fi

\end{picture}}
\endgroup



This means that a conditional statement is always
opened with an \mytt{:IF}. After the \mytt{IF} follows the
condition on which the statement acts.  After the
condition follows the instruction which is obeyed if
the condition is true.  It can be one of five words.

\renewcommand{\labelenumi}{(\roman{enumi})}
\begin{enumerate}
\def\theenumi{\roman{enumi}}

\item\myuline{\mytt{AND}}
\item\myuline{\mytt{OR}}

These allow us to have statements which depend on
the truth of two or more conditions.  After the \mytt{AND} or
\mytt{OR} the statement behaves as if there had just been an
\mytt{IF}.  This is indicated by the arrow in the statement
of the general conditional statement above.

Examples:-

\begin{addmargin}[0.1cm]{1em}%
\begin{lstlisting}
:IF(X<A+B)AND(Y)ZERO:AND(Z)ZERO:GOTO3
:IF(X<A:SIN)AND(Y)ZERO:OR(X)POSITIVE:REPEAT
:IF(X=Y:MOD(4))OR(Z:MOD<Y:MOD)THEN ... :,
\end{lstlisting}
\end{addmargin}

It is important to attach the correct meaning to
these multiple condition statements.  If the statement
consists of all \myuline{ORs} then the statement is true if any
\myuline{one} of the conditions is true.  Similarly if the
statement consists of all \myuline{ANDs} then the statement is
true only if \myuline{all} the conditions are true.  But the
statement (e.g.)

\begin{addmargin}[1cm]{1em}%
\begin{flushleft}
\mytt{:IF(\rlap{/}C)OR(\rlap{/}C')AND(\rlap{/}C'')OR(\rlap{/}C''')....GOTO(\rlap{/}S)}
\end{flushleft}
\end{addmargin}

\begin{flushleft}
would be interpreted as meaning
\end{flushleft}

\begin{addmargin}[0.3cm]{1em}%
\begin{math}
:IF(\mathrlap{/}C)OR\Bigl[(\mathrlap{/}C')AND
\Bigl\{(\mathrlap{/}C'')OR\Bigl((\mathrlap{/}C''')....
\Bigr)\Bigr\}\Bigr]GOTO(\mathrlap{/}S)
\end{math}
\end{addmargin}

The conditions can not be grouped in any other
way.  If possible, the combination of \mytt{AND}s and \mytt{OR}s
should be avoided, or at least, kept very simple.

\item\myuline{\mytt{REPEAT}}

If the statement is true obey next the instruction
immediately following the last unpaired \mytt{LOOP} instruction
(see LOOPS OF INSTRUCTIONS).

\begin{flushleft}
Example:—
\end{flushleft}

\begingroup
\centering
\makebox{%
\begin{picture}(70,35)
\thicklines

\put(15,28){\hbox{\kern3pt\mytt{:LOOP}}}
\put(15,19){\hbox{\kern3pt\mytt{:LOOP}}}
\put(15,9){\hbox{\kern3pt\mytt{:IF(X:=Y)REPEAT}}}
\put(15,1){\hbox{\kern3pt\mytt{:IF(X)ZERO:AND(Y)ZERO:REPEAT}}}

\put(16,2){\line(-1,0){5}}
\put(11,2){\line(0,1){27}}
\put(11,29){\vector(1,0){5}}

\put(16,10){\line(-1,0){3}}
\put(13,10){\line(0,1){10}}
\put(13,20){\vector(1,0){3}}

\end{picture}}
\endgroup

\item\myuline{\mytt{GOTO(\rlap{/}S)}}

If the statement is true obey next the instruction
immediately following the reference number with current
value \mytt{(\rlap{/}S)} (see JUMP SUFFIX).

Example:-

\begin{addmargin}[1cm]{1em}%
\begin{lstlisting}
:IF(X=3)GOTO6
:IF(Y<B:SQRT)AND(Y)NOT POSITIVE:GOTO(N1+4,ABC)
\end{lstlisting}
\end{addmargin}


\item\myuline{\mytt{THEN}}

\mytt{THEN} is always associated with a \mytt{;} (punched \mytt{:,})
or an \mytt{ELSE} and a \mytt{;} in the following manner:

\begin{enumerate}

\item
\begin{addmargin}[1cm]{1em}%
\begin{lstlisting}
:THEN , , , , :,
\end{lstlisting}
\end{addmargin}

\item
\begin{addmargin}[1cm]{1em}%
\begin{lstlisting}
:THEN , , , , :ELSE , , , , :,
\end{lstlisting}
\end{addmargin}

\end{enumerate}

The commas (,) indicate that there can be statements
between \mytt{THEN}, \mytt{ELSE} and \mytt{;}

\begin{enumerate}

\item If the statement is true then those statements
which occur between the \mytt{THEN} and the \mytt{;} are obeyed.  If
the statement is not true then control passes to the
statement immediately after the \mytt{;.} e.g.

\begin{addmargin}[0.1cm]{1em}%
\begin{lstlisting}
:IF(A<B*D)AND(C)ZERO:THEN , , , :,
\end{lstlisting}
\end{addmargin}

\item If the statement is true then those statements
which occur between the \mytt{THEN} and the \mytt{ELSE} are obeyed.
If the statement is not true then those statements
between the \mytt{ELSE} and the \mytt{;} are obeyed. e.g.

\begin{addmargin}[0.1cm]{1em}%
\begin{lstlisting}
:IF((A+B)*(A-B)-8)NEGA:THEN , , :ELSE , , :,
\end{lstlisting}
\end{addmargin}

\end{enumerate}

Any of the statements between \mytt{THEN}, \mytt{ELSE} and \mytt{;} can
themselves be \mytt{THEN} statements.  \myuline{This can occur to a
depth of 9}.

An \mytt{ELSE} is paired with the last impaired \mytt{THEN}.
A \mytt{;} is paired with the last unpaired \mytt{THEN} or \mytt{ELSE}.

Example:-

\begin{addmargin}[0.1cm]{1em}%
\begin{lstlisting}
:IF(A*A-(B*C) →P)NEGATIVE:THEN
-P:SQRT →Q,-1 →U
:ELSE
:IF(P)ZERO:THEN,0 →U:ELSE,P:SQRT →P,1 →U,:,
\end{lstlisting}
\end{addmargin}

\mytt{LOOP} or \mytt{DO} ..... \mytt{REPEAT} loops are quite distinct
from \mytt{THEN} statements and their parings are independent
so that we could have the following:

\begin{addmargin}[0.1cm]{1em}%
\begin{lstlisting}
(0 →N)LOOP:IF(A<BN)THEN, 1ON(N)REPEAT:,
\end{lstlisting}
\end{addmargin}

where the parings of \mytt{LOOP} ... \mytt{REPEAT} and \mytt{THEN} ... \mytt{;}
cross each other.  The above "compound" statement is
in fact a most useful table search.

\end{enumerate}

\begin{flushleft}
\myuline{Note}
\end{flushleft}

\begin{enumerate}
\item Before words \mytt{NEGATIVE} etc. and \mytt{AND} etc. CR, LF, Sp$^{2}$
are ignored.
\item In obeying a "compound" conditional statement as
soon as the truth or falsity of a statement is determined
then the rest of the conditions are "skipped" e.g. in

\begin{addmargin}[1cm]{1em}%
\begin{lstlisting}
:IF(A<B)OR(C<D)OR(X>Y:MOD)GOTO(6)
\end{lstlisting}
\end{addmargin}

If \mytt{(A<B)} is true then the whole statement is true and
the other conditions are not examined but e.g. in

\begin{addmargin}[1cm]{1em}%
\begin{lstlisting}
:IF(A<B)AND(C<D).....
\end{lstlisting}
\end{addmargin}

If \mytt{(A<B)} is true then because it is followed by \mytt{AND}
the other conditions must be examined.  However if
\mytt{(A<B)} is false then the statement is not true and the
other conditions are not examined.

\end{enumerate}

%\clearpage

\section{\myuline{\mytt{BRANCH}}}

\mytt{BRANCH} is a function of the accumulator and a
jump instruction which has up to 3 jump suffices.  It
transfers control according to the sign of the
accumulator. In general it is written:-

\begin{addmargin}[1cm]{1em}%
\vspace{0.2cm}
$\xi$\mytt{:BRANCH(R1,R2,R3)}
\end{addmargin}

\begin{flushleft}
where R = (\rlap{/}S) the general jump suffix (qv)
\end{flushleft}

It has the effect:— \\

\begin{tabular}{clcl}
If value of & $\xi <0$ & then control passes to & $R_{1}$ \\
.. & $\xi =0$ & .. & $R_{2}$ \\
.. & $\xi >0$ & .. & $R_{3}$ \\
\end{tabular}
\vspace{1ex}

The general case can be reduced to \\

\begin{addmargin}[1cm]{1em}%
%\vspace{0.2cm}
\begin{tabular}{ll}
   & $\xi$\mytt{:BRANCH(R1,R2)} \\
or & $\xi$\mytt{:BRANCH(R1)} \\
\end{tabular}
\end{addmargin}
\vspace{1em}

In these cases if the first one or two conditions
are not satisfied then no action is taken.
\vspace{1ex}

Examples:-

\begin{addmargin}[1cm]{1em}%
\begin{lstlisting}
(N-M)BRANCH(3, (N4+3), (N+1, ROUTINE))
\end{lstlisting}
\end{addmargin}

\begin{flushleft}
\myuline{Notes}
\end{flushleft}

\renewcommand{\labelenumi}{(\theenumi)}
\begin{enumerate}

\item If R consists of more than one "item" the brackets
\myuline{must} be included.  e.g. in the above example we must
have $(N_{4}+3)$.

In particular, in the reduced case we must write
(e.g.):-

\begin{addmargin}[1cm]{1em}%
\begin{lstlisting}
A:FRAC-W:BRANCH((N5-6))
\end{lstlisting}
\end{addmargin}

Note the two sets of brackets.
\end{enumerate}


\section{\myuline{\mytt{READ STATEMENTS}}}

On obeying an instruction such as:

\begin{addmargin}[1cm]{1em}%
\begin{lstlisting}
:READ(A,B,N,C)
\end{lstlisting}
\end{addmargin}

the first number is read off tape and sent to \mytt{A}, the
second to \mytt{B}, etc.

\mytt{A}, \mytt{B}, \mytt{N}, \mytt{C} can be variables or integers in any
order and there can any number of them.  They can
have simple or complex suffices

For form of punching of numbers to be input by the
\mytt{READ} see \myuline{FORM OF NUMBERS IN DATA}.

For effect of characters in searching for numbers
see table \ref{tbl:READ}

\begin{flushleft}
\myuline{Notes}
\end{flushleft}

\renewcommand{\labelenumi}{(\roman{enumi})}
\begin{enumerate}
\def\theenumi{\roman{enumi}}

\item\label{it:READARITH} In writing the \mytt{READ} instruction after the variable
which is to be read we can carry on with normal
arithmetic, with the restriction that the variable to
be read follows immediately after the \mytt{(} or \mytt{,} e.g. we
can write: (see (\ref{it:READAFT})).

\begin{addmargin}[1cm]{1em}%
\begin{lstlisting}
:READ(A,B3+C →X,Y'(P+Q))
\end{lstlisting}
\end{addmargin}

After the first comma; the number will be read,
sent to \mytt{B3}, \mytt{C} will be added and result sent to \mytt{X}.  We
can \myuline{not} write:

\begin{addmargin}[1cm]{1em}%
\begin{lstlisting}
:READ(A,(B+C)*X →Y)
\end{lstlisting}
\end{addmargin}

because the \mytt{B} does not
immediately follow the comma.

\item\label{it:READAFT} after \mytt{,} or \mytt{(} CR LF SpSp are ignored.
This is so that a \mytt{READ} instruction which is very long can be broken
at the end of a line of teleprinter page.

\end{enumerate}

\multicolinterrupt{
\noindent\begin{minipage}{\textwidth}
\captionof{table}{Effect of characters met in obeying \mytt{READ} instruction}\label{tbl:READ}
\begin{tabular}{|p{8cm}|p{8cm}|}
\hline
letter shift, figure shift    & \\
blank tape                    & are ignored \\
CR.\hspace{1em}  Sp Sp.\hspace{1em}  Sp.\hspace{1em}  LF\hspace{1em}  \mytt{=}  & \\
\hline
\mytt{. + -} digits           & introduces a number \\
\hline
\mytt{'}                      & introduces a title (terminated by \mytt{'}) which is copied \\
                              & onto the output tape directly. \\
\hline
\mytt{::}                     & Computer waits.  Continues after the last button of \\
                              & keyboard word is changed from its current position. \\
\hline
\mytt{)}                      & Computer stops. \\
\hline
All other characters          & Wrong.  Computer gives error signal. \\
\hline
\end{tabular}
\end{minipage}}

\clearpage

\section{\myuline{PRINT STATEMENTS}}

if m and n are integer constants there exist the
following print statements:-

\renewcommand{\labelenumi}{(\roman{enumi})}
\begin{enumerate}
\def\theenumi{\roman{enumi}}

\item\label{it:PRMN} $\xi_{v}$\mytt{:m.n} will print out the value of $\xi_{v}$ with
m digits before the decimal point
and n digits after.

\item\label{it:PRMS} $\xi_{v}$\mytt{:m/} ...... in floating point form with
m significant figures.

\item\label{it:PRS} $\xi_{v}$\mytt{:/} equivalent to \mytt{:8/}

\item\label{it:PRM} $\xi_{s}$\mytt{:m} ...... with m digits

\end{enumerate}

All numbers are preceded by a negate sign if
negative but by a space if not negative.  No terminating
symbol is punched after the number.  If the numbers
are to be re-input then it is necessary to punch the
terminating symbol by use of the title instruction
(q.v.).  All numbers are preceded by a figure shift.

Non-significant leading zeroes are replaced by
spaces.

\begin{flushleft}
In
\end{flushleft}

\begin{tabular}{llcclcc}
(\ref{it:PRMN}) & If m>12 or n>11 & then & method & (\ref{it:PRS}) & is & used. \\
(\ref{it:PRMS}) & If m>8 & .. & .. & (\ref{it:PRS}) & .. & .. \\
(\ref{it:PRM})  & \multicolumn{6}{l}{then print with m = 12.} \\
\end{tabular}

In (\ref{it:PRMN}) (\ref{it:PRMS}) (\ref{it:PRM})
if m is not large enough then an attempt is made
with m = m+1 until new m is big
enough or until the above restrictions apply.  The
number is \myuline{always} printed somehow.

Fractions are rounded off by the addition of
$-5 \times 10^{-n}$ to the number in (\ref{it:PRMN}) and $.5 \times 10^{-m}$ to the
mantissa in (\ref{it:PRMS}) (\ref{it:PRS}).

\begin{flushleft}
In
\end{flushleft}

\begin{tabular}{ll}
(\ref{it:PRMN}) m + n + 2 & characters are printed (+ shift) \\
(\ref{it:PRMS}) m + 6     & \hspace{3em} .. \\
(\ref{it:PRS}) 14         & \hspace{3em} .. \\
(\ref{it:PRM}) m + 2      & \hspace{3em} .. \\
\end{tabular}

\begin{flushleft}
\myuline{Note}
\end{flushleft}

\renewcommand{\labelenumi}{(\roman{enumi})}
\begin{enumerate}
\def\theenumi{\roman{enumi}}

\item The print instruction does \myuline{not} preserve the
accumulator and therefore can not occur in the middle
of an expression.

\item m and n \myuline{must} be integer constants e.g. \mytt{:6.3}
We can not write \mytt{:P.Q}.
\end{enumerate}

\
\section{\myuline{REFERENCE NUMBERS}}

It is desirable to be able to start obeying a
procedure or to break sequence in a procedure at a
point other than the first order.  For this reason we
\myuline{label} statements in the following manner, e.g.

\begin{addmargin}[1cm]{1em}%
\begin{lstlisting}
12) X+Y →Z
\end{lstlisting}
\end{addmargin}

That is, a label is an integer constant written with
a bracket before a statement. To avoid confusion there
is a restriction that the label can occur only at \myuline{the
beginning of a new line} on the teleprinter page.  A
label is also known as a reference number.

Control passes to the above instruction, say,
(which occurs e.g. in procedure \mytt{:ABC}) by orders of the
following types:

\setcounter{RowCounter}{0}
\begin{tabular}{rl}
\nextRow & \mytt{:IF(A)ZERO:GOTO(12)} \\
\nextRow & \mytt{:GOTO(12,ABC)} \\
\nextRow & \mytt{:ABC(12)} \\
\nextRow & \mytt{:START(12,ABC)} \\
\end{tabular}

\renewcommand{\labelenumi}{(\roman{enumi})}\begin{enumerate}
\def\theenumi{\roman{enumi}}

\item is a jump from a statement within the same
procedure (see CONDITIONAL STATEMENT)

\item is a jump from a statement within another
procedure (see UNCONDITIONAL JUMPS)

\item calls on \mytt{ABC} as a subroutine entering at 12
(see SUBROUTINE)

\item the whole programme is started at this point
(see START)
\end{enumerate}

A reference number can also be used in conjunction
with TRACE to give indication of the path of the
programme.

\begin{flushleft}
\myuline{Notes}
\end{flushleft}

\renewcommand{\labelenumi}{(\roman{enumi})}\begin{enumerate}
\def\theenumi{\roman{enumi}}

\item All instructions in the same procedure to which
reference is to be made must be given different
reference numbers.

\item Any reference numbers may occur, they need not
occur in sequence, and the numerical sequence of
references need not be complete.

\item The first order of a procedure is automatically
taken to be reference 0 (i.e. zero) but we must \myuline{not}
actually label it \mytt{0)}.

\item The highest reference number must be SET when the
procedure is SET at the beginning of the programme
(see SETTING OF PROCEDURES).
\end{enumerate}


\section{\myuline{BRACKETS OR DEPTH OF EXPRESSIONS}}

A pair of brackets \mytt{(\ \ )} in an expression or
statement gives the expression a \myuline{depth} of 1.  If there
is a pair of brackets within this pair then the
expression is of depth 2 ...etc.  There is a \myuline{maximum
allowable depth of 9}.

All brackets except those

\renewcommand{\labelenumi}{(\roman{enumi})}
\begin{enumerate}
\def\theenumi{\roman{enumi}}
\item around simple suffices
\item in \mytt{ON( )} instructions, and
\item in \mytt{BRANCH( )} instructions,
increase the depth of expression by 1
\end{enumerate}

\begin{flushleft}
Note that
\end{flushleft}

\begin{addmargin}[1cm]{1em}%
\begin{lstlisting}
:IF(A/(C*D+(E*F)))ZERO:GOTO1
\end{lstlisting}
\end{addmargin}

\begin{flushleft}
is of depth 3 but that
\end{flushleft}

\begin{addmargin}[1cm]{1em}%
\begin{lstlisting}
A/(B'(N+M)+C'(N-M)
\end{lstlisting}
\end{addmargin}

\begin{flushleft}
is of depth 2.
\end{flushleft}

That is, as the expression is read from the left
a \mytt{(} will increase the depth by 1 but the corresponding
\mytt{)} will decrease it accordingly.

\begin{flushleft}
\myuline{Notes}
\end{flushleft}

\renewcommand{\labelenumi}{(\theenumi)}
\begin{enumerate}

\item In arithmetic expressions the depth should be
kept to a minimum, disregarding for this purpose depths
due to brackets used merely for decoration.

\end{enumerate}

\begin{flushleft}
e.g. $y = d(a + bc)$ is more efficiently written
\end{flushleft}

\begin{addmargin}[1cm]{1em}%
\begin{lstlisting}
(B*C+A)*D →Y
\end{lstlisting}
\end{addmargin}

\begin{flushleft}
than
\end{flushleft}

\begin{addmargin}[1cm]{1em}%
\begin{lstlisting}
D*(A+(B*C)) →Y
\end{lstlisting}
\end{addmargin}

It can be seen in the first case all the
arithmetic is done directly in the accumulator but in
the second the accumulator has to be stored away twice
in order to perform the arithmetic.

It should be noted that what is defined as depth
in RDR/100/3.4 is not strictly what is defined as
depth here.


\section{\myuline{CONSTANT SEARCHING}}

In translation, copies of constants which occur in
arithmetic (but not \mytt{DATA} or \mytt{LIST}) are kept in a list.
If the same constant appears many times it is obviously
wasteful in space to have numerous copies of that
constant.  For this reason a facility is provided
for "constant searching".  In this a search is made
of constants already in the list before a new constant
is added.  Against this there is the fact that
searching can slow down translation considerably.
For this reason it is optional

If the \myuline{40} button in F1 on the number generator
is depressed constant searching will \myuline{not} take place
and vice versa.

As this button will normally be depressed constant
searching will not take place unless required.

Only the positive value of a constant is put in
the list.  Integers 1-27 inclusive are treated
separately and appear in a separate list in which
constant searching always takes place.


\section{\myuline{CHECK}}

The function of the accumulator known as CHECK
and written

\begin{addmargin}[1cm]{1em}%
\vspace{0.2cm}
\hspace{0.4cm}$\xi$\mytt{:*} \\
\end{addmargin}

is available in both fixed and floating point modes.
It provides optional print-out of intermediate results.
It is often very useful if CHECKS are included at
suitable points of programmes, to assist in detecting
programming errors, etc.

If B = 1 is set on the keyboard during the running
of a programme, CHECK will cause the contents of the
accumulator to be printed out on a new line preceded
by \mytt{*} and with the following print specifications:

variable as

\begin{addmargin}[1cm]{1em}%
\vspace{0.2cm}
\hspace{0.4cm}\mytt{:/} \\
\end{addmargin}

integer as

\begin{addmargin}[1cm]{1em}%
\vspace{0.2cm}
\hspace{0.4cm}\mytt{:1} \\
\end{addmargin}


CHECKS are only translated if B = 1 is set on
the keyboard during translation.  If B = 0, \mytt{:*} is
ignored.  Therefore we have two opportunities of
ignoring \mytt{:*}.  In translation and in running.  Obviously
if it is ignored in translation it can not he used at
all in running.  As a programme is run or translated
each \mytt{:*} is treated on its own merits, i.e. some may
be ignored, others assembled or used, depending on the
state of B at that moment.

\mytt{:*} does not affect the accumulator therefore it
can be used in the middle of arithmetic expressions.
e.g.

\begin{addmargin}[1cm]{1em}%
\begin{lstlisting}
((A+B):*:SQRT)*C →X
\end{lstlisting}
\end{addmargin}


\section{\myuline{TITLE AND OPTIONAL TITLE}}

\subsection{TITLE}

The statement which consists of a string of
characters enclosed by two inverted commas, is known
as a TITLE statement.  When obeyed the characters
between the inverted commas (dashes) will be printed
out on the output tape.

\begin{flushleft}
e.g.
\end{flushleft}

\begin{addmargin}[1cm]{1em}%
\begin{lstlisting}
'LBS/FT3'
\end{lstlisting}
\end{addmargin}

All characters are copied \myuline{except blank tape} which
is ignored.  \myuline{All} figure shifts or letter shifts whether
redundant or not are copied.  Every TITLE output is
preceded by figure shift.

It is important to note that a TITLE statement
is the only method (other than machine orders) by
which \myuline{any} characters can be output.  In particular, if
we want a number to be printed on a new line we mist
include a TITLE which consists of CRLF, i.e.

\begin{addmargin}[1cm]{1em}%
\begin{lstlisting}
'
'
\end{lstlisting}
\end{addmargin}

\begin{flushleft}
\myuline{Notes}
\end{flushleft}

\renewcommand{\labelenumi}{(\theenumi)}
\begin{enumerate}

\item If say n blanks are to be punched, this must be
done with the order
\end{enumerate}

\begin{addmargin}[1cm]{1em}%
\begin{lstlisting}
:DO(N)<740:):REPEAT
\end{lstlisting}
\end{addmargin}

\subsection{OPTIONAL TITLE}

This statement is similar to above except that
they are punched (e.g.)

\begin{addmargin}[1cm]{1em}%
\begin{lstlisting}
:'CONVERGE'
\end{lstlisting}
\end{addmargin}

i.e. with a colon before the normal title instruction.
These titles will only be output if the F2 40 button on
the keyboard is depressed in running.  They are always
input on translating.


\section{\myuline{TRACE}}

The TRACE is used to indicate which course a
programme is running.  This is useful if, for example,
the programme "gets lost”.  This it does by printing
out, on a new line, the value of a reference number
as control "passes through" it.  As reference numbers
are meaningless unless qualified by the procedure
which they are in the form of printing is (e.g.)

\begin{addmargin}[1cm]{1em}%
\begin{lstlisting}
13,ABC
\end{lstlisting}
\end{addmargin}

\begin{flushleft}
i.e. reference \mytt{13} of procedure \mytt{ABC}, As a procedure is
identified by the first 4 (or less) letters of a word
no mere than 4 letters are printed out.  The reference
number itself has the print specification
\end{flushleft}

\begin{addmargin}[1cm]{1em}%
\begin{lstlisting}
:4
\end{lstlisting}
\end{addmargin}

Similar to CHECK, the TRACE facility is accepted
in translation by depressing the F2 01 button on the
keyboard, and the TRACE is printed out on running by
depressing the same button.  As the TRACE facility
works in conjunction with reference numbers no extra
characters have to be punched on the programme tape.

\begin{flushleft}
\myuline{Note}
\end{flushleft}

Since the first order of a procedure is understood
to be reference 0 automatically, then on entering a
- procedure at the 1st order the TRACE would be e.g.

\begin{addmargin}[1cm]{1em}%
\begin{lstlisting}
0,ABC
\end{lstlisting}
\end{addmargin}


\section{\myuline{ARITHMETIC STATEMENT}}

Arithmetic is left \myuline{associative}, i.e. if we write
(in H-Code):~

\begin{addmargin}[1cm]{1em}%
\begin{lstlisting}
A+B*C →Y
\end{lstlisting}
\end{addmargin}

this is interpreted as meaning $y = (a+b)c$ and \myuline{not}
$y = a+(bc)$.  This is because the expression is
- interpreted as each operator is encountered - reading
from left to right.  No reference being made to the
next operator the normal convention of multiplication
and division being "stronger" than addition and
subtraction does not apply.

The standard functions of analysis (sin, sqrt,
etc.), are also left associative in the sense that they
are a function of the quantity on their left rather
than the right, e.g, the equation $y = sin(a+b)$ would
be written:-

\begin{addmargin}[1cm]{1em}%
\begin{lstlisting}
A+B:SIN →Y
\end{lstlisting}
\end{addmargin}

The \myuline{general arithmetic statement} is:—

\begin{addmargin}[1cm]{1em}%
\begin{lstlisting}
(-)A1:FCT£A2:FCT£A3:FCT...£AN:FCT
\end{lstlisting}
\end{addmargin}

where \mytt{A} is any variable or constant or \myuline{is itself an
arithmetic statement enclosed in brackets}

\myuline{FCT} is any standard algebraic function.

£ is any one of the operators \mytt{+ — * /} $\rightarrow$ but
$\rightarrow$A is not allowed if \mytt{A} is an expression in brackets
or a constant.
(see DEPTH OF EXPRESSIONS).

\begin{flushleft}
\myuline{Example}:—
\end{flushleft}

\begin{addmargin}[1cm]{1em}%
\begin{math}
i = ae^{-\frac{Rt}{2L}}\cos(wt+f)
\end{math}
\end{addmargin}
can be written:

\begin{addmargin}[1cm]{1em}%
\begin{lstlisting}
(W*T+F)COS*(-R*T*.5/L:EXP)*A →I
\end{lstlisting}
\end{addmargin}


\section{\myuline{ARITHMETIC OPERATIONS}}

The arithmetic operations

\begin{addmargin}[1cm]{1em}%
\mytt{+ - * /}
\end{addmargin}

\begin{flushleft}
are available in both fixed floating point
\end{flushleft}

\myuline{Integer multiplication} will give an incorrect
result if the result is greater than $2^{37}$ in absolute
value.  The error will not be detected if overflow
(see below) does \myuline{not} occur.

\myuline{Integer division} will give the correct answer if
the division has an exact result, if not, it will give
the largest integer not greater than the correct
answer.

\begin{addmargin}[1cm]{1em}%
\begin{lstlisting}
e.g. 6/2  =  3
    -6/2  = -3
     2/3  =  0
    15/4  =  3
  15/(-4) = -4
\end{lstlisting}
\end{addmargin}

\begin{flushleft}
\myuline{OVERFLOW} is said to take place if the result of an
arithmetic operation is outside the range of the machine
(see Form of Numbers)

\myuline{Floating-point overflow} causes the computer to stop
automatically.

\myuline{Fixed—point overflow} causes the computer to carry on
using the wrong result.

\myuline{Notes}
\end{flushleft}

\renewcommand{\labelenumi}{(\theenumi)}
\begin{enumerate}

\item Because of the nature of the translator direct
\myuline{division} by an integer which has a complex suffix
(q.v.) is not acceptable.

e.g. the term, where \mytt{N} \& \mytt{M} are integers

\begin{addmargin}[1cm]{1em}%
\begin{lstlisting}
N/M'(P+Q)
\end{lstlisting}
\end{addmargin}

will \myuline{not} be accepted.
If necessary, however, we
can write

\begin{addmargin}[1cm]{1em}%
\begin{lstlisting}
N/(M'(P+Q))
\end{lstlisting}
\end{addmargin}

\end{enumerate}


\section{\myuline{ALGEBRAIC FUNCTIONS}}

These functions occur anywhere inside arithmetic
statements.  The operand is in the accumulator and the
result is left in the accumulator.

\renewcommand{\labelenumi}{\alph{enumi})}
\begin{enumerate}
\def\theenumi{\alph{enumi}}

\newcounter{teMOD} % to hold a reference for later as nextRow does not support \label{}
\begin{minipage}{\columnwidth}
\item\label{it:AFBOTH} \myuline{Functions valid for both variables and integers}

\setcounter{RowCounter}{0}
\begin{tabular}{rll}
\nextRow & $\xi$\mytt{:SQ}  & Forms the square of $\xi$ \\
\nextRow & $\xi$\mytt{:NEG} & Negates $\xi$ \\
\nextRow\setcounter{teMOD}{\theRowCounter} & $\xi$\mytt{:MOD} & Forms $|\xi|$ i.e. the \\
  & & numerical value of $\xi$ \\
  & & e.g.  $|-3|=3, |0| = 0, |3|=3$. \\
\end{tabular}
\end{minipage}

\begin{minipage}{\columnwidth}
\item \myuline{Functions valid for floating-point variables only}

\setcounter{RowCounter}{0}
\begin{tabular}{rll}
\nextRow & $\xi_{v}$\mytt{:SQRT} & Forms $\sqrt{\xi}$ where $\xi \geqslant 0$ \\
\nextRow & $\xi_{v}$\mytt{:SIN}  & \multicolumn{1}{|l}{Form the trigonometrical} \\
\nextRow & $\xi_{v}$\mytt{:COS}  & \multicolumn{1}{|l}{functions of $\xi$ where $\xi$ is in} \\
\nextRow & $\xi_{v}$\mytt{:TAN}  & \multicolumn{1}{|l}{radians and $|\frac{\xi}{\pi}| < 2^{28}$} \\
\nextRow & $\xi_{v}$\mytt{:ARCTAN} & Forms the angle, in radians, \\
  & & whose tangent is $\xi$ such that it \\
  & & lies between $\pm\frac{\pi}{2}$ \\
\nextRow & $\xi_{v}$\mytt{:EXP}  & Forms the exponential of $\xi$, $e^{\xi}$, \\
  & & where $\xi \leqslant 254 \log_{e}2$ \\
\nextRow & $\xi_{v}$\mytt{:LOG}  & Forms the natural logarithm, \\
  & & $\log_{e}\xi$,  where $\xi > 0$ \\
\nextRow & $\xi_{v}$\mytt{:FRAC} & Forms the fractional part of \\
  & & such that $\Frac (\xi) + \Int (\xi) = \xi$ \\
  & & therefore e.g. $\Frac (3.4) = .4$, \\
  & & $\Frac (-5.4) = .6$, \\
\nextRow & $\xi_{v}$\mytt{:INT}  & Forms the integral part of $\xi$ and \\
  & & leaves it in integer form.  The \\
  & & integral part is the largest \\
  & & integer not greater than $\xi$. \\
  & & e.g. $\Int (3) = 3$, $\Int (3.4) = 3$, \\
  & & $\Int (-3.4) = -4$. \\
  & & \\
  & & \mytt{INT} is the only way of changing \\
  & & the mode of an expression from \\
  & & floating point to fixing point. \\
\end{tabular}
\end{minipage}

\begin{minipage}{\columnwidth}
\item \myuline{Functions valid for fixed-point variables only}

\setcounter{RowCounter}{0}
\begin{tabular}{rll}
\nextRow & $\xi_{s}$\mytt{:STAND} & Converts the integer value $\xi$ to \\
  or & $\xi_{s}$\mytt{:V} & to the floating-point variable of \\
  & & the same value, \mytt{STAND} is the only \\
  & & way of changing the mode of an \\
  & & expression from fixed point to \\
  & & floating-point. \\
\nextRow & $\xi_{s}$\mytt{:MOD(}$\xi_{s}'$\mytt{)} & Forms $\xi(\Modulo \xi')$ i.e. gives \\
  & & the remainder when $\xi$ is divided \\
  & & by $\xi'$.  $\xi' > 0$ and result $\geqslant 0$. \\
  & & e.g. 7(modulo 3) = 1, -7(modulo 3) = 2, \\
  & & $8(\Modulo 2) = 0$.  In fact, \\
  & & it performs the fixed-point \\
  & & operation $\xi-(\frac{\xi}{\xi'}*\xi')$.  As \mytt{MOD( )} \\
  & & is a subtraction rather than \\
  & & a division process the function \\
  & & should be used with care because \\
  & & if $\frac{\xi}{\xi'}$ is very large then the \\
  & & time of operation will become \\
  & & very long.  This function is \\
  & & distinguished from MOD in \ref{it:AFBOTH}) (\roman{teMOD}) \\
  & & above, with which it has no \\
  & & connection, by the following \\
  & & brackets. \\
\end{tabular}
\end{minipage}

\end{enumerate}

\begin{flushleft}
\myuline{Notes and Examples}
\end{flushleft}

\renewcommand{\labelenumi}{(\roman{enumi})}
\begin{enumerate}
\def\theenumi{\roman{enumi}}

\item the argument of \mytt{SIN}, \mytt{COS}, and \mytt{TAN} can be any angle
(subject to $|\frac{\xi}{\pi}| < 2^{28}$).  The \mytt{SIN} for example, is
taken as the angle modulo $2\pi$.

\item to find y say, the integral part of A, where y
is floating point we have to write

\begin{addmargin}[1cm]{1em}%
\begin{lstlisting}
A:INT:STAND →Y
\end{lstlisting}
\end{addmargin}

\begin{flushright}
(see (\ref{it:EIGHT}) below)
\end{flushright}

\item In order to raise a quantity to a power which is
not a whole number, e.g. to form

\begin{flushright}
$x = a^{b}$ we would write \hspace{5em}
\end{flushright}

\begin{addmargin}[1cm]{1em}%
\begin{lstlisting}
A:LOG*B:EXP →X
\end{lstlisting}
\end{addmargin}

\item In order to save time in computing, it is often
possible to avoid calculating a function more times
than necessary, for example, to programme the equation:-

\begin{addmargin}[1cm]{1em}%
\begin{math}
y = \frac{e^{x}+e^{-x}}{2}
\end{math}
\end{addmargin}

\begin{flushleft}
it is very easy to write:-
\end{flushleft}

\begin{addmargin}[1cm]{1em}%
\begin{lstlisting}
((X:EXP)+(-X:EXP))/2 →Y
\end{lstlisting}
\end{addmargin}

\begin{flushleft}
but more economical to write:
\end{flushleft}

\begin{addmargin}[1cm]{1em}%
\begin{lstlisting}
((X:EXP →Y)*(1.0/Y))/2 →Y
\end{lstlisting}
\end{addmargin}

\begin{flushleft}
Note the use of $y$ as an intermediate store for $e^{x}$, which
which avoids calculating an exponential more often than necessary.
\end{flushleft}

\item The equation $y=e^{-(x-a)^{2}}$ can be written either

\begin{addmargin}[1em]{1em}%
\begin{lstlisting}
     (-((X-A)SQ))EXP →Y
or   (X-A)SQ:NEG:EXP →Y
\end{lstlisting}
\end{addmargin}

\begin{flushleft}
It is generally better to use \mytt{NEG} when negating an
expression \mytt{-(....)} involves increasing the depth
of the arithmetic which is to be avoided if possible.
\end{flushleft}

\item it is useful to remember that $\Int (A) + \Frac (A) = A$.
In fact finding $\Frac (A)$ the computer finds $A - \Int (A)$.
Therefore if both frac and int are needed it is better to calculate only int.

\item Functions to calculate arcsin and arccos are not provided
but:-

\begin{addmargin}[1cm]{1em}%
\begin{math}
\arcsin x = \arctan(\frac{x}{\sqrt{1-x^{2}}} \\
\arccos x = \arctan(\frac{\sqrt{1-x^{2}}}{x}
\end{math}
\end{addmargin}

e.g. for the arcsin of A we could write:-

\begin{addmargin}[1cm]{1em}%
\begin{lstlisting}
A/(1.0-(A*A):SQRT):ARCTAN
\end{lstlisting}
\end{addmargin}

\item\label{it:EIGHT} If we want to form $y = n + x$ when $x$ and $y$ are in
floating point form and $n$ is an integer we would write:-

\begin{addmargin}[1cm]{1em}%
\begin{lstlisting}
N:STAND+X →Y
\end{lstlisting}
\end{addmargin}

\item \mytt{MOD( )} can be used to depth.  e.g. we could write:-

\begin{addmargin}[1cm]{1em}%
\begin{lstlisting}
N:MOD(M:MOD(L)+2)
\end{lstlisting}
\end{addmargin}

\end{enumerate}


\section{\myuline{$\alpha$-$\theta$ STORES}}

$\alpha$ is a floating point variable which, as well as
possessing all the properties of other floating point
variables, has the special property that if we write:-

\begin{addmargin}[1cm]{1em}%
\begin{math}
\{\xi_{v}, \xi_{v}^{(i)}, \xi_{v}^{(ii)}, \xi_{v}^{(iii)}, \ldots \}
\end{math}
\end{addmargin}

then this is equivalent to writing

\begin{addmargin}[1cm]{1em}%
\begin{tabular}{ll}
   & $\xi_{v} \rightarrow \alpha$ \\
   & $\xi_{v}^{(i)} \rightarrow \alpha_{1}$ \\
   & $\xi_{v}^{(ii)} \rightarrow \alpha_{2}$ \\
   & $\xi_{v}^{(iii)} \rightarrow \alpha_{3}$ \\
   & $\vdots$ \\
or in general \hspace{4em}  & $\xi_{v}^{(n)} \rightarrow \alpha_{n}$ \\
for n = 0, 1, 2..... & \\
\end{tabular}
\end{addmargin}

Similarly $\theta$ is a fixed point variable such that

\begin{addmargin}[1cm]{1em}%
\begin{math}
[\xi_{s}, \xi_{s}^{(i)}, \xi_{s}^{(ii)}, \xi_{s}^{(iii)}, \ldots ]
\end{math}
\end{addmargin}

is equivalent to $\xi_{s}^{(n)} \rightarrow \theta_{n}$, for n = 0, 1, 2...

In general we can write (e.g.)

\begin{addmargin}[1cm]{1em}%
\begin{math}
\{\xi_{v}, \xi_{v}^{(i)}; \xi_{s}; \xi_{v}^{(ii)}, \xi_{v}^{(iii)}; \xi_{s}^{(i)}, \ldots ]
\end{math}
\end{addmargin}

or

\begin{addmargin}[1cm]{1em}%
\begin{math}
[\xi_{s}; \xi_{v}, \xi_{v}^{(i)}; \xi_{s}^{(i)}, \xi_{s}^{(ii)}, \ldots ]
\end{math}
\end{addmargin}

which are equivalent to

\begin{addmargin}[1cm]{1em}%
$\xi_{v}^{(n)} \rightarrow \alpha_{n}$ for n = 0, 1, 2 .... \\
$\xi_{s}^{(m)} \rightarrow \theta_{m}$ for m = 0, 1, 2 .... \\
\end{addmargin}

That is, the type of bracket determines the mode of the
first expression, thereafter the mode is changed by \mytt{;}
(Although this indication of mode by type of bracket
is not strictly necessary, it is required as a safeguard
against careless programming, in particular against
writing an integer in floating point without the
decimal point).

The reverse procedure is (e.g.)

\begin{addmargin}[1cm]{1em}%
\begin{math}
\rightarrow(A,A^{(i)},N,A^{(ii)},N^{(i)}, \ldots)
\end{math}
\end{addmargin}

where A are floating and N are fixed-point variables
with simple or complex suffices is equivalent to:—

\begin{addmargin}[1cm]{1em}%
$\alpha_{n} \rightarrow A^{n}$ for n = 0, 1, 2 .... \\
$\theta_{m} \rightarrow N^{m}$ for m = 0, 1, 2 .... \\
\end{addmargin}

Variables and integers can be mixed in any order
and semi-colons between them are not necessary although
they can be used if desired.

\begin{flushleft}
\myuline{Notes}
\end{flushleft}

\renewcommand{\labelenumi}{(\roman{enumi})}
\begin{enumerate}
\def\theenumi{\roman{enumi}}

\item In the $\rightarrow( )$ instruction as in the \mytt{READ}
instruction (Note (\ref{it:READARITH})) after the variable we can carry
on with normal arithmetic.

\item $\alpha$, $\theta$ are set automatically by \mytt{SETV}, \mytt{SETS} to be
$\alpha_{9}$, $\theta_{9}$.  But if more used these must be set.

\item These $\alpha$-$\theta$ statements ("nests") imply comma, \myuline{except}
that after the final \mytt{)} a letter introduces a word.
this is so we can write (e.g.)

\begin{addmargin}[1cm]{1em}%
\begin{lstlisting}
{A,B+C;N}ABC
\end{lstlisting}
\end{addmargin}

where \mytt{ABC} is a word statement (usually a defined
function).  This is the most common use of a nesting
store.

\begin{addmargin}[1cm]{1em}%
\begin{tabular}{lcl}
\mytt{\{\ \ \ \}} & is punched & \mytt{*(\ \ \ )} \\
\mytt{[\ \ \ ]} & .. & \mytt{:(\ \ \ )} \\
\mytt{;} & .. & \mytt{:,} \\
  & & \\
\multicolumn{3}{l}{$\alpha$ if the teleprinter character \mytt{@}} \\
\multicolumn{3}{l}{$\theta$ is letter O.} \\
\end{tabular}
\end{addmargin}

\item \myuline{Example} If there was a defined procedure \mytt{QUAD}
which gave the two roots of a quadratic given the three
coefficients the order to do this might look:—

\begin{addmargin}[1cm]{1em}%
\begin{lstlisting}
*(1.6,3.6,-4.53)QUAD →(X1,X2)
\end{lstlisting}
\end{addmargin}

\end{enumerate}


\section{\myuline{BAR A $\bar{A}$}}

$\bar{A}$, called BAR A, punched \mytt{=A}, is an \myuline{integer} which
has the value

Absolute Address (A) -1

where A is any variable or integer.

If B, for example, has been set as

\begin{addmargin}[1cm]{1em}%
\mytt{:LV(B:1)} or \mytt{:LS(B:1)}
\end{addmargin}

then if $\bar{A}$ has been sent to \mytt{N} then

\mytt{BN}  will be equivalent to \mytt{A}

This means that subroutines can be written in
terms of any variables irrespective of what variables a
master procedure will use.  Part of the input data to
a subroutine would be say, $\bar{A}$, telling the subroutine
that such and such a parameter was A.

Arithmetic can be performed with $\bar{A}$ etc. but we
\myuline{cannot} write:—

\begin{addmargin}[1cm]{1em}%
\begin{lstlisting}
=A-=B
\end{lstlisting}
\end{addmargin}

\begin{flushleft}
it must be
\end{flushleft}

\begin{addmargin}[1cm]{1em}%
\begin{lstlisting}
=A-(=B)
\end{lstlisting}
\end{addmargin}

$\bar{A}$ cannot have a suffix, but if we wanted the
address of $A_{3}$ we would write

\begin{addmargin}[1cm]{1em}%
\begin{lstlisting}
=A+3
\end{lstlisting}
\end{addmargin}


\section{\myuline{SUNDRY INSTRUCTIONS IN BODY OF PROCEDURE}}

\hspace{\parindent}\myuline{\mytt{STOP}}

On obeying this order dynamically, the machine
will STOP.  There is no way of carrying on from that
point.

\myuline{\mytt{WAIT}}

On obeying this order dynamically the machine will
WAIT, until the Address 2: 1 button is changed (see
KEYBOARD).  The machine will then obey the next order
etc.

\myuline{\mytt{CLOSE}}

Signifies that the procedure is closed.

\myuline{\mytt{EXIT}}

On obeying this order dynamically control will
pass back to the master procedure which is using the
current procedure as a subroutine (see USE OF
PROCEDURES)

\myuline{Line-feed}

In translation \myuline{line-feed} has the same significance
as comma which closes a statement.  After a line-feed
all subsequent lf, cr, $sp^{2}$ are ignored until the next
significant character.

\myuline{Comma} \mytt{,}

In translation, closes a statement (see STRUCTURE
OF PROCEDURE).

\myuline{\mytt{DATA}}

Accepts a data tape in the form of \mytt{LIST}.

\myuline{\mytt{:} CR LF}

Ignored in position of \mytt{+} in arithmetic statement

\myuline{\mytt{::}}

In translation the computer will \myuline{wait} - until the
address 2:1 button is changed (see KEYBOARD).  The
machine will then carry on translating.


\section{\myuline{TELEPRINTER CHARACTER: LAYOUT OF PROGRAMME ON TELEPRINTER SHEETS}}

Generally speaking except for misuse of carriage
return without line-feed, if a programme 'looks right'
on the teleprinter print-out then the tape which
produced it will be correct also-and vice-versa.  The
following notes on the characters are true for all parts
of the programme-procedures, data etc. expect\footnote{maybe should be ``except''} those
character appearing between \mytt{'\ \ \ '} in title sequences.

\myuline{blank-tape} is always ignored.

\myuline{Figure-shift} \\
\indent\myuline{Letter-shift} have no significance other than as
shift characters.  The use of more shift characters
than necessary is allowed.  Hence letter shift (ls) can
be used as an erase character, by backspacing and over-punching
the incorrect character with ls - then figure-shift (fs) if necessary.

\myuline{?} is never used

\myuline{single space} (Sp) is ignored everywhere but ...

\myuline{double space} (Spsp or $Sp^{2}$) has significance as a
terminating character to numbers, words etc.  It has the
same properties as carriage return (cr)

\myuline{@} is understood to be $\alpha$

\myuline{letter O} when used as an integer variable is
taken to be theta ($\theta$).

\myuline{Title characters} - i.e. those between \mytt{'\ \ \ '} are
treated independently. Notes:—

\myuline{blank-tape} In a straight copy from input to
output is copied but when a title is stored and output
in running bl is ignored.

\myuline{Shifts} Are copied \myuline{exactly} as they appear on the
input tape.

\myuline{\mytt{'}} Cannot be used in a title sequence.

\begin{tabular}{ll|l|l}
Elliott Characters & \mytt{£}  &               & $\rightarrow$ \\
                   & \mytt{\$} & correspond to & \mytt{<} \\
                   & \mytt{\%} &               & \mytt{>} \\
\end{tabular}


\section{\myuline{WORDS}}

A word can consist of 1, 2, 3, or 4 letters.  If
a word has 4 letters then extra letters can be added to
the end if required - e.g. The word for repeat is
\mytt{REPE} but we write this as \mytt{REPEAT} to make the printout
more readable.  These extra letters have no significance
whatsoever.  The translator in reading a word will take
note of up to the first 4 letters and keep on reading,
but ignoring, letter until the first non-alphabet
character, which is taken to belong to the next
operation.

To distinguish words from variables, all words are
preceded by colon e.g.

\begin{addmargin}[1cm]{1em}%
\begin{lstlisting}
:WORD
\end{lstlisting}
\end{addmargin}

\begin{flushleft}
\myuline{but} where there is \myuline{no ambiguity} i.e. in a position
where a letter can only introduce a word, this colon
can be omitted e.g.

\begin{tabular}{lccl}
\mytt{:GOTO(1,ABC)} & instead & of & \mytt{:GOTO(1,:ABC)} \\
\mytt{(A+B)SIN} & .. & .. & \mytt{(A+B):SIN} \\
but we must have & \multicolumn{2}{r}{\mytt{A:SIN}} & \\[1ex]
\mytt{:DEFINE(ABC)} & .. & .. & \mytt{:DEFINE(:ABC)} \\[1ex]
\multicolumn{4}{l}{or \myuline{any initial word}} \\[1ex]
\mytt{SET} & .. & .. & \mytt{:SET} \\[1ex]
\end{tabular}

But \myuline{if in doubt always put in the colon}
\end{flushleft}

Throughout this manual the colon has been omitted
where there is no ambiguity both in the English and in
the examples of the coding.


\section{\myuline{SUMMARY OF FACILITIES WITHIN PROCEDURE}}
See Table \ref{tbl:PROCFAC}


\section{\myuline{USE OF COMPUTER KEYBOARD}}
See Table \ref{tbl:WORDGEN}

%\clearpage

\multicolinterrupt{
\noindent\begin{minipage}{\textwidth}
\captionof{table}{Summary of facilities within procedure}\label{tbl:PROCFAC}
\begin{tabular}{|l|lr|}
\hline
\mytt{DEFINE( )........CLOSE}    & Defines body of procedure & \\
\hline
\mytt{SQ NEG MOD ON(...)}$\star$ & Algebraic function of variable or integer \hspace{2cm} & \\
\hline
\mytt{SQRT SIN COS TAN EXP ARCTAN LOG FRAC INT} \hspace{2cm} & Algebraic function of variable only. & \\
\hline
\mytt{STAND MOD (..)}            & Algebraic function of integer only. & \\
\hline
\mytt{BRANCH( , , )}             & Jump on sign of LHS    & $\star$ \\
\hline
\mytt{:n.m :m/ :m :/}            & Prints    & $\star$ \\
\hline
\mytt{:*}                        & Check    & \\
\hline
Any defined function \mytt{ABC}  & Defined function    & $\star$ \\
\hline
\mytt{LOOP....REPEAT} \hspace{2em} \mytt{DO( )....REPEAT} & LOOPS    & $\star$ \\
\hline
\mytt{GOTO}                      &    & $\star$ \\
\hline
\mytt{\ \ \ \ \ \ \ \ GOTO}      &    & $\star$ \\
\mytt{IF\ ....\ REPEAT}          & Conditional statement    & \\
\mytt{\ \ \ \ \ \ \ \ AND OR}    &    &  \\
\hline
\mytt{GOTO EXIT}                 & Unconditional jumps    & $\star$ \\
\hline
\mytt{IF .... THEN .... ELSE ... ;} & Then Clause    & $\star$ \\
\hline
\mytt{READ DATA}                 & Input of numbers    & $\star$ \\
\hline
\mytt{STOP WATT}                 & Running stops    & $\star$ \\
\hline
\mytt{[\ \ ]\ \ \ \{\ \ \}\ \ \ →(\ \ )}  & $\alpha$,$\theta$ stores    & $\star$ \\
\hline
\mytt{3)}                        & Reference number    & $\star$ \\
\hline
\mytt{'\ \ \ '}\hspace{4em}\mytt{:'\ \ \ '} & Titles    & $\star$ \\
\hline
\mytt{<\ \ \ )}                   & Machine-code block    & $\star$ \\
\hline
LF \mytt{,}                      & Close statement    & \\
\hline
\end{tabular}

\vspace{0.3cm}
$\star$ Those instructions imply comma.

\vspace{0.3cm}
\begin{flushleft}
\myuline{Note}
\end{flushleft}

Although a defined function implies a comma, if it is immediately followed by a title statement
without a comma to split them then this will be an error as the translator takes the
\mytt{'} as the introduction to a complex suffix, even though this is not allowed.

\end{minipage}}%multicolinterrupt

\vfill
\ \\
\pagebreak[4]

% Elliott 803 Word Generator (Keyboard)
\multicolinterrupt{
\noindent\begin{minipage}{\textwidth}
\captionof{table}{Use of Computer Keyboard}\label{tbl:WORDGEN}

\newcommand{\button}[1]{\parbox{1.3em}{\centering O \\ #1}}
\newcommand{\fbutton}[1]{\fbox{\parbox{1.3em}{\centering O \\ #1}}}
\setlength{\fboxsep}{2pt}

\begin{tabular}{|*{14}C|}
\hline
\multicolumn{14}{|l|}{FUNCTION 1} \\
\multicolumn{14}{|l|}{ } \\
\fbutton{$4_{1}$} & \button{$2_{1}$} & \button{$1_{1}$} & \button{$4_{2}$} & \button{$2_{2}$} & \button{$1_{2}$} &
\multicolumn{8}{r|}{\parbox{3cm}{Buttons and number \\ in square are used}} \\
\multicolumn{14}{|l|}{ } \\

\multicolumn{14}{|l|}{ADDRESS 1} \\
\multicolumn{14}{|l|}{ } \\
\button{4096} & \button{2048} & \button{1024} & \fbutton{512} & \button{256} & \button{128} &
\button{64} & \button{32} & \button{16} & \button{8} & \button{4} & \button{2} & \fbutton{1} & \fbutton{B} \\
\multicolumn{14}{|l|}{ } \\

\multicolumn{14}{|l|}{FUNCTION 2} \\
\multicolumn{14}{|l|}{ } \\
\fbutton{$4_{1}$} & \button{$2_{1}$} & \button{$1_{1}$} & \button{$4_{2}$} & \button{$2_{2}$} & \fbutton{$1_{2}$} &
\multicolumn{8}{r|}{ } \\
\multicolumn{14}{|l|}{ } \\

\multicolumn{14}{|l|}{ADDRESS 2} \\
\multicolumn{14}{|l|}{ } \\
\button{4096} & \button{2048} & \button{1024} & \button{512} & \button{256} & \button{128} &
\button{64} & \button{32} & \button{16} & \button{8} & \button{4} & \button{2} & \fbutton{1} &  \\
\multicolumn{14}{|l|}{ } \\

\hline
\multicolumn{4}{|l}{\myuline{OPERATION}} & \multicolumn{3}{|l}{BUTTON} & \multicolumn{3}{|l}{DOWN} & \multicolumn{4}{|l|}{UP} \\
\hline
\multicolumn{4}{|l}{\myuline{INPUTTING BINARY TAPE}} & \multicolumn{3}{|l}{Function 1:$4_{1}$} & \multicolumn{3}{|l}{Input programme} & \multicolumn{4}{|l|}{Compare programme} \\
\hline
\multicolumn{4}{|l}{\myuline{INPUTTING NORMALLY}} & \multicolumn{3}{|l}{Function 1:$4_{1}$} & \multicolumn{3}{|l}{No constant search} & \multicolumn{4}{|l|}{Constant search} \\
\multicolumn{4}{|l}{} & \multicolumn{3}{|l}{Address 1:512} & \multicolumn{3}{|l}{Entry to translate} & \multicolumn{4}{|l|}{} \\
\multicolumn{4}{|l}{} & \multicolumn{3}{|l}{Address 1:512} & \multicolumn{3}{|l}{Entry to obey START} & \multicolumn{4}{|l|}{} \\
\multicolumn{4}{|l}{} & \multicolumn{2}{|r}{and :1} & & \multicolumn{3}{|c}{i.e. 513} & \multicolumn{4}{|l|}{} \\
\multicolumn{4}{|l}{} & \multicolumn{3}{|l}{B} & \multicolumn{3}{|l}{Translate check} & \multicolumn{4}{|l|}{Ignore check} \\
\multicolumn{4}{|l}{} & \multicolumn{3}{|l}{Function 2:$1_{2}$} & \multicolumn{3}{|l}{Translate Trace} & \multicolumn{4}{|l|}{Ignore trace} \\

\hline
\multicolumn{4}{|l}{\myuline{RUNNING}} & \multicolumn{3}{|l}{B} & \multicolumn{3}{|l}{Print Check} & \multicolumn{4}{|l|}{Ignore check} \\
\multicolumn{4}{|l}{} & \multicolumn{3}{|l}{Function 2:$4_{1}$} & \multicolumn{3}{|l}{Print optional Title} & \multicolumn{4}{|l|}{Ignore Option Title} \\
\multicolumn{4}{|l}{} & \multicolumn{3}{|l}{Function 2:$1_{2}$} & \multicolumn{3}{|l}{Print Trace} & \multicolumn{4}{|l|}{Ignore Trace} \\
\hline
\multicolumn{4}{|l}{\myuline{GENERAL}} & \multicolumn{3}{|l}{Address 2:1} & \multicolumn{3}{|l}{Stop waiting if \myuline{up}} & \multicolumn{4}{|l|}{Stop waiting if \myuline{down}} \\
\hline
\end{tabular}

% to encourage clearpage to work
\vspace{12cm}
\ \\
\end{minipage}}%multicolinterrupt

\clearpage
%\pagebreak[4]

\section{\myuline{ERROR INDICATION}}

In translation and in running errors will be
indicated by the continuous output of a letter.
Although this letter will give some indication of the
type of error encountered it should not be looked upon
as a rigorous classification.  A particular indication
can be very misleading due to the very nature of an
error. The following indications will be given.

\renewcommand{\labelenumi}{(\roman{enumi})}
\begin{enumerate}
\def\theenumi{\roman{enumi}}

\item \myuline{In Translation}

\myuline{Continuous letter—shift-letter}


\begin{tabular}{cl}

\myuline{Indication} & \myuline{Classification} \\
A & Incompatibility of variables and \\
  & integers. Variables not set. \\
B & Incorrect characters at beginning of \\
  & arithmetic expression \\
C & Incorrect character elsewhere in \\
  & arithmetic expression \\
D & Error in data or number in body of \\
  & programme \\
E & WORD wrongly used or not set \\
F & \\
G & \\
H & Error in DATA read \\
I & Reference too big or set twice. \\
  & Depth of loops, arith. too deep. \\
  & Programme too big. \\
J & Too many digits in integers, various \\
  & odd mistakes \\
K & Excess of left or right-hand brackets \\
  & in arith. depth of loop at CLOSE not \\
  & zero. \\
\end{tabular}

\item \myuline{In Running}

\myuline{continuous letter-shift-letter}

\begin{tabular}{cl}
A & Jumping to reference which has not \\
  & been set or calling on procedure \\
  & which has not been defined. \\
\end{tabular}

\myuline{Continuous digit (or letter)}

\begin{tabular}{cl}
1 or A & A in sin A, cos A or tan A too big \\
       & (see appropriate section) \\
2 or B & $A\leqslant 0$ in log A \\
3 or S & $A>254 \log_{e} 2$ in exp A \\
5 or U & $A<0$ in SQRT A \\
\end{tabular}
\end{enumerate}

\renewcommand{\labelenumi}{\ifnum\value{enumi}=1 \myuline{Notes} (\roman{enumi})\else (\roman{enumi})\fi}
\begin{enumerate}
\def\theenumi{\roman{enumi}}

\item In translation if the computer 'gets lost'
this is most probably because on a :: wait after a SET
instruction the translator has been entered at the
beginning instead of the wait being 'cancelled'.

\item In running if the programme does strange
unaccountable things, this is most probably because the
variable SET's have not been high enough and the
programme is being overwritten by data.
\end{enumerate}


\begin{framed}
\begin{flushleft}
\myuline{\textit{Hand written notes:}}
\end{flushleft}

\myuline{Continuous letter-shift blank}

\begin{tabular}{cl}
       & DO(<1) \\
\end{tabular}


\myuline{Modification}

\begin{addmargin}[1cm]{1em}%
\begin{lstlisting}
:I → Contents of accumulator
\end{lstlisting}
\end{addmargin}
\end{framed}

% to encourage all text in one column
\ifTwoColumn\columnbreak\fi
\vspace{6cm}
\ \\

\ifTwoColumn
\end{multicols}
\fi

% ===== End Page ==================================================

\end{document}
